\documentclass[a4paper]{article}

\usepackage[margin=1in]{geometry}
\usepackage[utf8]{inputenc}
\usepackage[T1]{fontenc}
\usepackage{microtype}
%\usepackage{longtable}
\usepackage{mathtools}
\usepackage{multicol}
\usepackage{multirow}
\usepackage{booktabs}
\usepackage{courier}
\usepackage{listings}
\usepackage{enumitem}
\usepackage{mdwlist} % tighter description environment (starred)
\usepackage{gensymb}
\usepackage{hyperref}
\usepackage{fp}
\usepackage{longtable}
\usepackage{sparklines}

%% \usepackage{showframe}  %% Use this for debugging margin overruns in tables.
\usepackage{adjustbox}
\usepackage{graphicx}
\usepackage{softdev}
\usepackage{amsmath}
\usepackage{pifont}
\usepackage{xspace}
\usepackage{pdflscape}
\usepackage{sparklines}
\usepackage{float}
\usepackage{siunitx}


\newcommand{\krun}{\textsc{Krun}\xspace}
\newcommand{\graalce}{\textsc{Graal CE}\xspace}
\newcommand{\graalcehs}{\textsc{Graal CE Hotspot}\xspace}
\newcommand{\bencherseven}{Linux$_\mathrm{1240v5}$\xspace}
\newcommand{\bencherten}{Linux$_\mathrm{1240v6}$\xspace}
\AtBeginDocument{
\newlength{\blankheight}
\settototalheight{\blankheight}{
$\begin{array}{rr}
\scriptstyle{0.16} \\[-6pt]
\scriptscriptstyle{\pm0.000}
\end{array}$
}
}

%
% Sparklines.
%
\DeclareRobustCommand{\flatc}{%
\setlength{\sparklinethickness}{0.4pt}%
\begin{sparkline}{1.5}
\spark 0.0 0.35
       1.0 0.35
       /%
\end{sparkline}\xspace}
\DeclareRobustCommand{\nosteadystate}{%
\setlength{\sparklinethickness}{0.4pt}%
\begin{sparkline}{1.5}
\spark 0.0 0.35
       0.1 0.5
       0.3 0.2
       0.5 0.5
       0.7 0.2
       0.9 0.5
       1.0 0.35
       /%
\end{sparkline}\xspace}
\DeclareRobustCommand{\warmup}{%
\setlength{\sparklinethickness}{0.4pt}%
\begin{sparkline}{1.5}
\spark 0.0 0.8
       0.5 0.8
       0.5 0.0
       1.0 0.0
       /%
\end{sparkline}\xspace}
\DeclareRobustCommand{\slowdown}{%
\setlength{\sparklinethickness}{0.4pt}%
\begin{sparkline}{1.5}
\spark 0.0 0.0
       0.5 0.0
       0.5 0.8
       1.0 0.8
       /%
\end{sparkline}\xspace}
\DeclareRobustCommand{\badinconsistent}{%
\setlength{\sparklinethickness}{0.4pt}%
\begin{sparkline}{1.5}
\spark 0.1 0.4
       0.9 0.4
       /%
\spark 0.1 0.2
       0.9 0.2
       /%
\spark 0.1 0.6
       0.9 0.0
       /%
\spark 0.1 0.0
       0.9 0.6
       /%
\end{sparkline}\xspace}
\DeclareRobustCommand{\goodinconsistent}{%
\setlength{\sparklinethickness}{0.4pt}%
\begin{sparkline}{1.5}
\spark 0.1 0.4
       0.9 0.4
       /%
\spark 0.1 0.2
       0.9 0.2
       /%
\end{sparkline}\xspace}

%
% Colours.
%
\definecolor{lightred}{HTML}{e88a8a}
\definecolor{lightyellow}{HTML}{e8e58a}
\definecolor{lightgreen}{HTML}{8ae89c}


\begin{document}


\title{Analysing the Renaissance Benchmark Suite~\footnote{Updates to this paper will be found at \url{XXX}}}

\author{Edd Barrett and Laurence Tratt}

\maketitle

\begin{abstract}
\noindent In this early draft we evaluate the warmup characteristics of the
Renaissance Benchmark Suite.~\cite{prokopec19renaissance} We do so using the
benchmarking practices from our earlier work on VM
warmup~\cite{barrett16warmup}. Namely, we run the benchmarks for longer than is
typical and we use a controlled benchmarking environment enforced by our
benchmark runner, \krun. We find that XXX.
\end{abstract}


\section{Benchmarking Method}
\label{sec:eval}

Following the precident of the Renaissance suite, we will run the benchmarks
using two different \emph{VM configurations}: \graalce and \graalcehs.
\edd{what are these?}. Each VM configuration and benchmark pairing is run 10
times using a fresh invocation and thus OS process. We call each such
invocation a \emph{process executions}. Within each process execution the
benchmark is repeated 2000 times. We call these inner repetitions
\emph{in-process iterations}. In contrast, the Renaissance authors run
XXX\edd{how many?}

We run the Renaissance benchmarks in a similar fashion to that used in our
warmup experiment.~\cite{barrett16warmup}. We use the \krun runner with
settings designed to mimimise (where possible) measurement variation introduced
by the benchmarking environment. For example, consistent stack and heap limits
are set, undesirable daemons are disabled during benchmarking, the system
temperature before each run is not allowed deviate too much, and network
interfaces are disabled during benchmarking.

One small change was made to \krun. For each process execution, usually a
\krun experiment will call an \emph{iterations runner} which (amongst other
things) allocates the results array in an appropriate manner and collects
wallclock times using an appropriate high-resolution monotonic clock. Instead
of using an iterations runner, we changed \krun, adding the ability to defer
results collection to an external program.~\footnote{See
\texttt{ExternalSuiteVMDef} in the \krun source code} Although such an approach
means that we lose some of the advanced \krun functionality (e.g.
measuring core cycles and checking \texttt{APERF}/\texttt{MPERF} ratios), this
does allow us to re-use a large chunk of the benchmark running code from the
Renaissance suite.

\edd{which clock do they use? Talk about it?}

We ran the experiments on two similar machines. The first, \bencherseven, is a
Dell PowerEdge R330 with an Intel Xeon E3-1240 v5 CPU, running at 3.50GHz and
with 24GiB of RAM. The second machine, \bencherten, is also a Dell PowerEdge
R330, but has an Intel Xeon E3-1240 v6 CPU running at 3.70GHz
and with 32Gib of RAM. Both machines run Debian-9.9 with an indentical set of
installed packages. Both machines have hyper-threading and turbo boost
disabled.

\section{Results}
\label{sec:results}

\edd{tables don't break properly when placed in a table env...}
{
\setlength\sparkspikewidth{1.5pt}
\definecolor{sparkbottomlinecolor}{gray}{0.4}
% Older versions of sparklines do not expose bottomlinethickness
\renewcommand{\sparkbottomline}[1][1]{\pgfsetlinewidth{0.2pt}%
  \color{sparkbottomlinecolor}%
  \pgfline{\pgfxy(0,0)}{\pgfxy(#1,0)}\color{sparklinecolor}}

\begin{longtable}{ll@{\hspace{0cm}}ll@{\hspace{-1cm}}r@{\hspace{0cm}}r@{\hspace{0cm}}r@{\hspace{0cm}}l@{\hspace{.3cm}}ll@{\hspace{-1cm}}r@{\hspace{0cm}}r@{\hspace{0cm}}r}
\multicolumn{1}{c}{\multirow{2}{*}{}}&&&\multicolumn{1}{c}{} &\multicolumn{1}{c}{Steady}&\multicolumn{1}{c}{Steady}&\multicolumn{1}{c}{Steady}\\&&&\multicolumn{1}{c}{Class.} &\multicolumn{1}{c}{iter (\#)} &\multicolumn{1}{c}{iter (s)}&\multicolumn{1}{c}{perf (s)} \\\hline
\endhead
akka-uct&\begin{minipage}[c][\blankheight]{0pt}\end{minipage}&\multirow{20}{*}{\rotatebox[origin=c]{90}{graal-ce}}&\multicolumn{1}{l}{\badinconsistent \scriptsize($6$\flatc, $2$\warmup, $2$\nosteadystate)}&\begin{minipage}[c][\blankheight]{0pt}\end{minipage}&\begin{minipage}[c][\blankheight]{0pt}\end{minipage}&\begin{minipage}[c][\blankheight]{0pt}\end{minipage}\\ 
als&\begin{minipage}[c][\blankheight]{0pt}\end{minipage}&&\multicolumn{1}{l}{\goodinconsistent \scriptsize($8$\warmup, $2$\flatc)}&$
\begin{array}{c}
\scriptstyle{68.5} \\[-6pt]
\scriptscriptstyle{(1.0, 165.5)}
\end{array}
$
\noindent\parbox[p]{4ex}{\renewcommand{\sparklineheight}{2.75}
\begin{sparkline}{4}
\sparkspike 0.10 0.20
\sparkspike 0.20 0.00
\sparkspike 0.30 0.00
\definecolor{sparkspikecolor}{named}{red}
\sparkspike 0.40 0.30
\definecolor{sparkspikecolor}{named}{black}
\sparkspike 0.50 0.20
\sparkspike 0.60 0.00
\sparkspike 0.70 0.00
\sparkspike 0.80 0.00
\sparkspike 0.90 0.10
\sparkspike 1.00 0.20
\sparkbottomline
\end{sparkline}
\renewcommand{\sparklineheight}{1.75}}
&$
\begin{array}{c}
\scriptstyle{160.04} \\[-6pt]
\scriptscriptstyle{(0.000, 377.898)}
\end{array}
$
\noindent\parbox[p]{4ex}{\renewcommand{\sparklineheight}{2.75}
\begin{sparkline}{4}
\sparkspike 0.10 0.20
\sparkspike 0.20 0.00
\sparkspike 0.30 0.00
\definecolor{sparkspikecolor}{named}{red}
\sparkspike 0.40 0.30
\definecolor{sparkspikecolor}{named}{black}
\sparkspike 0.50 0.10
\sparkspike 0.60 0.10
\sparkspike 0.70 0.00
\sparkspike 0.80 0.00
\sparkspike 0.90 0.10
\sparkspike 1.00 0.20
\sparkbottomline
\end{sparkline}
\renewcommand{\sparklineheight}{1.75}}
&$
\begin{array}{c}
\scriptstyle{2.19283} \\[-6pt]
\scriptscriptstyle{\pm0.049554}
\end{array}
$
\noindent\parbox[p]{4ex}{\renewcommand{\sparklineheight}{2.75}
\begin{sparkline}{4}
\sparkspike 0.10 0.10
\sparkspike 0.20 0.00
\sparkspike 0.30 0.20
\definecolor{sparkspikecolor}{named}{red}
\sparkspike 0.40 0.20
\definecolor{sparkspikecolor}{named}{black}
\sparkspike 0.50 0.20
\sparkspike 0.60 0.10
\sparkspike 0.70 0.00
\sparkspike 0.80 0.10
\sparkspike 0.90 0.00
\sparkspike 1.00 0.10
\sparkbottomline
\end{sparkline}
\renewcommand{\sparklineheight}{1.75}}
\\ 
chi-square&\begin{minipage}[c][\blankheight]{0pt}\end{minipage}&&\multicolumn{1}{l}{\badinconsistent \scriptsize($8$\warmup, $2$\slowdown)}&$
\begin{array}{c}
\scriptstyle{51.0} \\[-6pt]
\scriptscriptstyle{(43.0, 1010.3)}
\end{array}
$
\noindent\parbox[p]{4ex}{\renewcommand{\sparklineheight}{2.75}
\begin{sparkline}{4}
\definecolor{sparkspikecolor}{named}{red}
\sparkspike 0.10 0.60
\definecolor{sparkspikecolor}{named}{black}
\sparkspike 0.20 0.20
\sparkspike 0.30 0.00
\sparkspike 0.40 0.00
\sparkspike 0.50 0.10
\sparkspike 0.60 0.00
\sparkspike 0.70 0.00
\sparkspike 0.80 0.00
\sparkspike 0.90 0.00
\sparkspike 1.00 0.10
\sparkbottomline
\end{sparkline}
\renewcommand{\sparklineheight}{1.75}}
&$
\begin{array}{c}
\scriptstyle{47.13} \\[-6pt]
\scriptscriptstyle{(39.761, 822.993)}
\end{array}
$
\noindent\parbox[p]{4ex}{\renewcommand{\sparklineheight}{2.75}
\begin{sparkline}{4}
\definecolor{sparkspikecolor}{named}{red}
\sparkspike 0.10 0.60
\definecolor{sparkspikecolor}{named}{black}
\sparkspike 0.20 0.20
\sparkspike 0.30 0.00
\sparkspike 0.40 0.00
\sparkspike 0.50 0.10
\sparkspike 0.60 0.00
\sparkspike 0.70 0.00
\sparkspike 0.80 0.00
\sparkspike 0.90 0.00
\sparkspike 1.00 0.10
\sparkbottomline
\end{sparkline}
\renewcommand{\sparklineheight}{1.75}}
&$
\begin{array}{c}
\scriptstyle{0.84468} \\[-6pt]
\scriptscriptstyle{\pm0.075056}
\end{array}
$
\noindent\parbox[p]{4ex}{\renewcommand{\sparklineheight}{2.75}
\begin{sparkline}{4}
\sparkspike 0.10 0.40
\sparkspike 0.20 0.00
\definecolor{sparkspikecolor}{named}{red}
\sparkspike 0.30 0.20
\definecolor{sparkspikecolor}{named}{black}
\sparkspike 0.40 0.00
\sparkspike 0.50 0.10
\sparkspike 0.60 0.10
\sparkspike 0.70 0.00
\sparkspike 0.80 0.00
\sparkspike 0.90 0.00
\sparkspike 1.00 0.20
\sparkbottomline
\end{sparkline}
\renewcommand{\sparklineheight}{1.75}}
\\ 
db-shootout&\begin{minipage}[c][\blankheight]{0pt}\end{minipage}&&\multicolumn{1}{l}{\flatc}&\begin{minipage}[c][\blankheight]{0pt}\end{minipage}&\begin{minipage}[c][\blankheight]{0pt}\end{minipage}&$
\begin{array}{c}
\scriptstyle{5.30899} \\[-6pt]
\scriptscriptstyle{\pm0.169694}
\end{array}
$
\noindent\parbox[p]{4ex}{\renewcommand{\sparklineheight}{2.75}
\begin{sparkline}{4}
\sparkspike 0.10 0.10
\sparkspike 0.20 0.00
\sparkspike 0.30 0.00
\sparkspike 0.40 0.10
\sparkspike 0.50 0.00
\sparkspike 0.60 0.00
\sparkspike 0.70 0.00
\sparkspike 0.80 0.20
\definecolor{sparkspikecolor}{named}{red}
\sparkspike 0.90 0.40
\definecolor{sparkspikecolor}{named}{black}
\sparkspike 1.00 0.20
\sparkbottomline
\end{sparkline}
\renewcommand{\sparklineheight}{1.75}}
\\ 
dec-tree&\begin{minipage}[c][\blankheight]{0pt}\end{minipage}&&\multicolumn{1}{l}{\warmup}&$
\begin{array}{c}
\scriptstyle{1177.0} \\[-6pt]
\scriptscriptstyle{(1131.5, 1224.5)}
\end{array}
$
\noindent\parbox[p]{4ex}{\renewcommand{\sparklineheight}{2.75}
\begin{sparkline}{4}
\sparkspike 0.10 0.20
\sparkspike 0.20 0.10
\sparkspike 0.30 0.10
\sparkspike 0.40 0.00
\definecolor{sparkspikecolor}{named}{red}
\sparkspike 0.50 0.10
\definecolor{sparkspikecolor}{named}{black}
\sparkspike 0.60 0.10
\sparkspike 0.70 0.00
\sparkspike 0.80 0.00
\sparkspike 0.90 0.30
\sparkspike 1.00 0.10
\sparkbottomline
\end{sparkline}
\renewcommand{\sparklineheight}{1.75}}
&$
\begin{array}{c}
\scriptstyle{2013.98} \\[-6pt]
\scriptscriptstyle{(1966.775, 2036.056)}
\end{array}
$
\noindent\parbox[p]{4ex}{\renewcommand{\sparklineheight}{2.75}
\begin{sparkline}{4}
\sparkspike 0.10 0.20
\sparkspike 0.20 0.00
\sparkspike 0.30 0.10
\sparkspike 0.40 0.00
\sparkspike 0.50 0.00
\sparkspike 0.60 0.10
\definecolor{sparkspikecolor}{named}{red}
\sparkspike 0.70 0.30
\definecolor{sparkspikecolor}{named}{black}
\sparkspike 0.80 0.20
\sparkspike 0.90 0.00
\sparkspike 1.00 0.10
\sparkbottomline
\end{sparkline}
\renewcommand{\sparklineheight}{1.75}}
&$
\begin{array}{c}
\scriptstyle{1.67633} \\[-6pt]
\scriptscriptstyle{\pm0.069302}
\end{array}
$
\noindent\parbox[p]{4ex}{\renewcommand{\sparklineheight}{2.75}
\begin{sparkline}{4}
\sparkspike 0.10 0.20
\sparkspike 0.20 0.10
\definecolor{sparkspikecolor}{named}{red}
\sparkspike 0.30 0.20
\definecolor{sparkspikecolor}{named}{black}
\sparkspike 0.40 0.00
\sparkspike 0.50 0.10
\sparkspike 0.60 0.10
\sparkspike 0.70 0.00
\sparkspike 0.80 0.10
\sparkspike 0.90 0.00
\sparkspike 1.00 0.20
\sparkbottomline
\end{sparkline}
\renewcommand{\sparklineheight}{1.75}}
\\ 
dotty&\begin{minipage}[c][\blankheight]{0pt}\end{minipage}&&\multicolumn{1}{l}{\warmup}&$
\begin{array}{c}
\scriptstyle{191.5} \\[-6pt]
\scriptscriptstyle{(151.0, 299.2)}
\end{array}
$
\noindent\parbox[p]{4ex}{\renewcommand{\sparklineheight}{2.75}
\begin{sparkline}{4}
\sparkspike 0.10 0.40
\sparkspike 0.20 0.00
\definecolor{sparkspikecolor}{named}{red}
\sparkspike 0.30 0.20
\definecolor{sparkspikecolor}{named}{black}
\sparkspike 0.40 0.00
\sparkspike 0.50 0.10
\sparkspike 0.60 0.00
\sparkspike 0.70 0.10
\sparkspike 0.80 0.10
\sparkspike 0.90 0.00
\sparkspike 1.00 0.10
\sparkbottomline
\end{sparkline}
\renewcommand{\sparklineheight}{1.75}}
&$
\begin{array}{c}
\scriptstyle{231.72} \\[-6pt]
\scriptscriptstyle{(187.467, 354.169)}
\end{array}
$
\noindent\parbox[p]{4ex}{\renewcommand{\sparklineheight}{2.75}
\begin{sparkline}{4}
\sparkspike 0.10 0.40
\sparkspike 0.20 0.00
\definecolor{sparkspikecolor}{named}{red}
\sparkspike 0.30 0.20
\definecolor{sparkspikecolor}{named}{black}
\sparkspike 0.40 0.00
\sparkspike 0.50 0.10
\sparkspike 0.60 0.00
\sparkspike 0.70 0.10
\sparkspike 0.80 0.00
\sparkspike 0.90 0.10
\sparkspike 1.00 0.10
\sparkbottomline
\end{sparkline}
\renewcommand{\sparklineheight}{1.75}}
&$
\begin{array}{c}
\scriptstyle{1.10616} \\[-6pt]
\scriptscriptstyle{\pm0.008107}
\end{array}
$
\noindent\parbox[p]{4ex}{\renewcommand{\sparklineheight}{2.75}
\begin{sparkline}{4}
\sparkspike 0.10 0.20
\sparkspike 0.20 0.00
\sparkspike 0.30 0.10
\sparkspike 0.40 0.00
\definecolor{sparkspikecolor}{named}{red}
\sparkspike 0.50 0.20
\definecolor{sparkspikecolor}{named}{black}
\sparkspike 0.60 0.00
\sparkspike 0.70 0.20
\sparkspike 0.80 0.00
\sparkspike 0.90 0.10
\sparkspike 1.00 0.20
\sparkbottomline
\end{sparkline}
\renewcommand{\sparklineheight}{1.75}}
\\ 
fj-kmeans&\begin{minipage}[c][\blankheight]{0pt}\end{minipage}&&\multicolumn{1}{l}{\goodinconsistent \scriptsize($5$\flatc, $5$\warmup)}&$
\begin{array}{c}
\scriptstyle{1.5} \\[-6pt]
\scriptscriptstyle{(1.0, 3.1)}
\end{array}
$
\noindent\parbox[p]{4ex}{\renewcommand{\sparklineheight}{2.75}
\begin{sparkline}{4}
\definecolor{sparkspikecolor}{named}{red}
\sparkspike 0.10 0.50
\definecolor{sparkspikecolor}{named}{black}
\sparkspike 0.20 0.00
\sparkspike 0.30 0.00
\sparkspike 0.40 0.40
\sparkspike 0.50 0.00
\sparkspike 0.60 0.00
\sparkspike 0.70 0.00
\sparkspike 0.80 0.00
\sparkspike 0.90 0.00
\sparkspike 1.00 0.10
\sparkbottomline
\end{sparkline}
\renewcommand{\sparklineheight}{1.75}}
&$
\begin{array}{c}
\scriptstyle{2.40} \\[-6pt]
\scriptscriptstyle{(0.000, 10.181)}
\end{array}
$
\noindent\parbox[p]{4ex}{\renewcommand{\sparklineheight}{2.75}
\begin{sparkline}{4}
\definecolor{sparkspikecolor}{named}{red}
\sparkspike 0.10 0.50
\definecolor{sparkspikecolor}{named}{black}
\sparkspike 0.20 0.00
\sparkspike 0.30 0.00
\sparkspike 0.40 0.40
\sparkspike 0.50 0.00
\sparkspike 0.60 0.00
\sparkspike 0.70 0.00
\sparkspike 0.80 0.00
\sparkspike 0.90 0.00
\sparkspike 1.00 0.10
\sparkbottomline
\end{sparkline}
\renewcommand{\sparklineheight}{1.75}}
&$
\begin{array}{c}
\scriptstyle{4.45894} \\[-6pt]
\scriptscriptstyle{\pm0.531239}
\end{array}
$
\noindent\parbox[p]{4ex}{\renewcommand{\sparklineheight}{2.75}
\begin{sparkline}{4}
\sparkspike 0.10 0.30
\sparkspike 0.20 0.00
\sparkspike 0.30 0.00
\sparkspike 0.40 0.10
\definecolor{sparkspikecolor}{named}{red}
\sparkspike 0.50 0.10
\definecolor{sparkspikecolor}{named}{black}
\sparkspike 0.60 0.20
\sparkspike 0.70 0.00
\sparkspike 0.80 0.10
\sparkspike 0.90 0.00
\sparkspike 1.00 0.20
\sparkbottomline
\end{sparkline}
\renewcommand{\sparklineheight}{1.75}}
\\ 
future-genetic&\begin{minipage}[c][\blankheight]{0pt}\end{minipage}&&\multicolumn{1}{l}{\badinconsistent \scriptsize($5$\warmup, $4$\slowdown, $1$\flatc)}&$
\begin{array}{c}
\scriptstyle{13.0} \\[-6pt]
\scriptscriptstyle{(4.2, 15.0)}
\end{array}
$
\noindent\parbox[p]{4ex}{\renewcommand{\sparklineheight}{2.75}
\begin{sparkline}{4}
\sparkspike 0.10 0.10
\sparkspike 0.20 0.00
\sparkspike 0.30 0.00
\sparkspike 0.40 0.00
\sparkspike 0.50 0.00
\sparkspike 0.60 0.10
\sparkspike 0.70 0.10
\sparkspike 0.80 0.10
\definecolor{sparkspikecolor}{named}{red}
\sparkspike 0.90 0.20
\definecolor{sparkspikecolor}{named}{black}
\sparkspike 1.00 0.40
\sparkbottomline
\end{sparkline}
\renewcommand{\sparklineheight}{1.75}}
&$
\begin{array}{c}
\scriptstyle{19.01} \\[-6pt]
\scriptscriptstyle{(4.963, 22.752)}
\end{array}
$
\noindent\parbox[p]{4ex}{\renewcommand{\sparklineheight}{2.75}
\begin{sparkline}{4}
\sparkspike 0.10 0.10
\sparkspike 0.20 0.00
\sparkspike 0.30 0.00
\sparkspike 0.40 0.00
\sparkspike 0.50 0.10
\sparkspike 0.60 0.00
\sparkspike 0.70 0.10
\sparkspike 0.80 0.10
\definecolor{sparkspikecolor}{named}{red}
\sparkspike 0.90 0.20
\definecolor{sparkspikecolor}{named}{black}
\sparkspike 1.00 0.40
\sparkbottomline
\end{sparkline}
\renewcommand{\sparklineheight}{1.75}}
&$
\begin{array}{c}
\scriptstyle{1.59950} \\[-6pt]
\scriptscriptstyle{\pm0.098687}
\end{array}
$
\noindent\parbox[p]{4ex}{\renewcommand{\sparklineheight}{2.75}
\begin{sparkline}{4}
\sparkspike 0.10 0.10
\sparkspike 0.20 0.00
\sparkspike 0.30 0.00
\sparkspike 0.40 0.00
\sparkspike 0.50 0.10
\sparkspike 0.60 0.10
\sparkspike 0.70 0.10
\definecolor{sparkspikecolor}{named}{red}
\sparkspike 0.80 0.10
\definecolor{sparkspikecolor}{named}{black}
\sparkspike 0.90 0.10
\sparkspike 1.00 0.40
\sparkbottomline
\end{sparkline}
\renewcommand{\sparklineheight}{1.75}}
\\ 
gauss-mix&\begin{minipage}[c][\blankheight]{0pt}\end{minipage}&&\multicolumn{1}{l}{\warmup}&$
\begin{array}{c}
\scriptstyle{126.0} \\[-6pt]
\scriptscriptstyle{(82.0, 244.7)}
\end{array}
$
\noindent\parbox[p]{4ex}{\renewcommand{\sparklineheight}{2.75}
\begin{sparkline}{4}
\definecolor{sparkspikecolor}{named}{red}
\sparkspike 0.10 0.50
\definecolor{sparkspikecolor}{named}{black}
\sparkspike 0.20 0.00
\sparkspike 0.30 0.00
\sparkspike 0.40 0.20
\sparkspike 0.50 0.10
\sparkspike 0.60 0.10
\sparkspike 0.70 0.00
\sparkspike 0.80 0.00
\sparkspike 0.90 0.00
\sparkspike 1.00 0.10
\sparkbottomline
\end{sparkline}
\renewcommand{\sparklineheight}{1.75}}
&$
\begin{array}{c}
\scriptstyle{116.07} \\[-6pt]
\scriptscriptstyle{(71.162, 220.603)}
\end{array}
$
\noindent\parbox[p]{4ex}{\renewcommand{\sparklineheight}{2.75}
\begin{sparkline}{4}
\sparkspike 0.10 0.40
\definecolor{sparkspikecolor}{named}{red}
\sparkspike 0.20 0.10
\definecolor{sparkspikecolor}{named}{black}
\sparkspike 0.30 0.00
\sparkspike 0.40 0.20
\sparkspike 0.50 0.00
\sparkspike 0.60 0.20
\sparkspike 0.70 0.00
\sparkspike 0.80 0.00
\sparkspike 0.90 0.00
\sparkspike 1.00 0.10
\sparkbottomline
\end{sparkline}
\renewcommand{\sparklineheight}{1.75}}
&$
\begin{array}{c}
\scriptstyle{0.87829} \\[-6pt]
\scriptscriptstyle{\pm0.055511}
\end{array}
$
\noindent\parbox[p]{4ex}{\renewcommand{\sparklineheight}{2.75}
\begin{sparkline}{4}
\sparkspike 0.10 0.10
\sparkspike 0.20 0.00
\sparkspike 0.30 0.00
\sparkspike 0.40 0.00
\sparkspike 0.50 0.00
\sparkspike 0.60 0.00
\sparkspike 0.70 0.00
\sparkspike 0.80 0.10
\definecolor{sparkspikecolor}{named}{red}
\sparkspike 0.90 0.50
\definecolor{sparkspikecolor}{named}{black}
\sparkspike 1.00 0.30
\sparkbottomline
\end{sparkline}
\renewcommand{\sparklineheight}{1.75}}
\\ 
log-regression&\begin{minipage}[c][\blankheight]{0pt}\end{minipage}&&\multicolumn{1}{l}{\badinconsistent \scriptsize($6$\nosteadystate, $2$\warmup, $2$\slowdown)}&\begin{minipage}[c][\blankheight]{0pt}\end{minipage}&\begin{minipage}[c][\blankheight]{0pt}\end{minipage}&\begin{minipage}[c][\blankheight]{0pt}\end{minipage}\\ 
mnemonics&\begin{minipage}[c][\blankheight]{0pt}\end{minipage}&&\multicolumn{1}{l}{\badinconsistent \scriptsize($8$\warmup, $2$\slowdown)}&$
\begin{array}{c}
\scriptstyle{3.0} \\[-6pt]
\scriptscriptstyle{(3.0, 323.8)}
\end{array}
$
\noindent\parbox[p]{4ex}{\renewcommand{\sparklineheight}{2.75}
\begin{sparkline}{4}
\definecolor{sparkspikecolor}{named}{red}
\sparkspike 0.10 0.80
\definecolor{sparkspikecolor}{named}{black}
\sparkspike 0.20 0.00
\sparkspike 0.30 0.00
\sparkspike 0.40 0.00
\sparkspike 0.50 0.00
\sparkspike 0.60 0.00
\sparkspike 0.70 0.00
\sparkspike 0.80 0.00
\sparkspike 0.90 0.00
\sparkspike 1.00 0.20
\sparkbottomline
\end{sparkline}
\renewcommand{\sparklineheight}{1.75}}
&$
\begin{array}{c}
\scriptstyle{7.64} \\[-6pt]
\scriptscriptstyle{(7.334, 1072.700)}
\end{array}
$
\noindent\parbox[p]{4ex}{\renewcommand{\sparklineheight}{2.75}
\begin{sparkline}{4}
\definecolor{sparkspikecolor}{named}{red}
\sparkspike 0.10 0.80
\definecolor{sparkspikecolor}{named}{black}
\sparkspike 0.20 0.00
\sparkspike 0.30 0.00
\sparkspike 0.40 0.00
\sparkspike 0.50 0.00
\sparkspike 0.60 0.00
\sparkspike 0.70 0.00
\sparkspike 0.80 0.00
\sparkspike 0.90 0.00
\sparkspike 1.00 0.20
\sparkbottomline
\end{sparkline}
\renewcommand{\sparklineheight}{1.75}}
&$
\begin{array}{c}
\scriptstyle{3.31763} \\[-6pt]
\scriptscriptstyle{\pm0.090074}
\end{array}
$
\noindent\parbox[p]{4ex}{\renewcommand{\sparklineheight}{2.75}
\begin{sparkline}{4}
\sparkspike 0.10 0.10
\sparkspike 0.20 0.00
\sparkspike 0.30 0.10
\sparkspike 0.40 0.10
\sparkspike 0.50 0.10
\definecolor{sparkspikecolor}{named}{red}
\sparkspike 0.60 0.20
\definecolor{sparkspikecolor}{named}{black}
\sparkspike 0.70 0.10
\sparkspike 0.80 0.00
\sparkspike 0.90 0.20
\sparkspike 1.00 0.10
\sparkbottomline
\end{sparkline}
\renewcommand{\sparklineheight}{1.75}}
\\ 
naive-bayes&\begin{minipage}[c][\blankheight]{0pt}\end{minipage}&&\multicolumn{1}{l}{\badinconsistent \scriptsize($8$\warmup, $2$\nosteadystate)}&\begin{minipage}[c][\blankheight]{0pt}\end{minipage}&\begin{minipage}[c][\blankheight]{0pt}\end{minipage}&\begin{minipage}[c][\blankheight]{0pt}\end{minipage}\\ 
neo4j-analytics&\begin{minipage}[c][\blankheight]{0pt}\end{minipage}&&\multicolumn{1}{l}{\badinconsistent \scriptsize($5$\warmup, $4$\flatc, $1$\nosteadystate)}&\begin{minipage}[c][\blankheight]{0pt}\end{minipage}&\begin{minipage}[c][\blankheight]{0pt}\end{minipage}&\begin{minipage}[c][\blankheight]{0pt}\end{minipage}\\ 
par-mnemonics&\begin{minipage}[c][\blankheight]{0pt}\end{minipage}&&\multicolumn{1}{l}{\badinconsistent \scriptsize($9$\warmup, $1$\slowdown)}&$
\begin{array}{c}
\scriptstyle{5.0} \\[-6pt]
\scriptscriptstyle{(3.5, 293.0)}
\end{array}
$
\noindent\parbox[p]{4ex}{\renewcommand{\sparklineheight}{2.75}
\begin{sparkline}{4}
\definecolor{sparkspikecolor}{named}{red}
\sparkspike 0.10 0.80
\definecolor{sparkspikecolor}{named}{black}
\sparkspike 0.20 0.00
\sparkspike 0.30 0.00
\sparkspike 0.40 0.00
\sparkspike 0.50 0.00
\sparkspike 0.60 0.00
\sparkspike 0.70 0.00
\sparkspike 0.80 0.00
\sparkspike 0.90 0.00
\sparkspike 1.00 0.20
\sparkbottomline
\end{sparkline}
\renewcommand{\sparklineheight}{1.75}}
&$
\begin{array}{c}
\scriptstyle{11.94} \\[-6pt]
\scriptscriptstyle{(7.561, 807.503)}
\end{array}
$
\noindent\parbox[p]{4ex}{\renewcommand{\sparklineheight}{2.75}
\begin{sparkline}{4}
\definecolor{sparkspikecolor}{named}{red}
\sparkspike 0.10 0.80
\definecolor{sparkspikecolor}{named}{black}
\sparkspike 0.20 0.00
\sparkspike 0.30 0.00
\sparkspike 0.40 0.00
\sparkspike 0.50 0.00
\sparkspike 0.60 0.00
\sparkspike 0.70 0.00
\sparkspike 0.80 0.00
\sparkspike 0.90 0.00
\sparkspike 1.00 0.20
\sparkbottomline
\end{sparkline}
\renewcommand{\sparklineheight}{1.75}}
&$
\begin{array}{c}
\scriptstyle{2.82552} \\[-6pt]
\scriptscriptstyle{\pm0.084534}
\end{array}
$
\noindent\parbox[p]{4ex}{\renewcommand{\sparklineheight}{2.75}
\begin{sparkline}{4}
\sparkspike 0.10 0.10
\sparkspike 0.20 0.00
\sparkspike 0.30 0.10
\sparkspike 0.40 0.20
\definecolor{sparkspikecolor}{named}{red}
\sparkspike 0.50 0.20
\definecolor{sparkspikecolor}{named}{black}
\sparkspike 0.60 0.10
\sparkspike 0.70 0.00
\sparkspike 0.80 0.00
\sparkspike 0.90 0.10
\sparkspike 1.00 0.20
\sparkbottomline
\end{sparkline}
\renewcommand{\sparklineheight}{1.75}}
\\ 
philosophers&\begin{minipage}[c][\blankheight]{0pt}\end{minipage}&&\multicolumn{1}{l}{\badinconsistent \scriptsize($8$\slowdown, $2$\nosteadystate)}&\begin{minipage}[c][\blankheight]{0pt}\end{minipage}&\begin{minipage}[c][\blankheight]{0pt}\end{minipage}&\begin{minipage}[c][\blankheight]{0pt}\end{minipage}\\ 
reactors&\begin{minipage}[c][\blankheight]{0pt}\end{minipage}&&\multicolumn{1}{l}{\badinconsistent \scriptsize($7$\flatc, $3$\slowdown)}&$
\begin{array}{c}
\scriptstyle{1.0} \\[-6pt]
\scriptscriptstyle{(1.0, 708.9)}
\end{array}
$
\noindent\parbox[p]{4ex}{\renewcommand{\sparklineheight}{2.75}
\begin{sparkline}{4}
\definecolor{sparkspikecolor}{named}{red}
\sparkspike 0.10 0.70
\definecolor{sparkspikecolor}{named}{black}
\sparkspike 0.20 0.00
\sparkspike 0.30 0.00
\sparkspike 0.40 0.20
\sparkspike 0.50 0.00
\sparkspike 0.60 0.00
\sparkspike 0.70 0.00
\sparkspike 0.80 0.00
\sparkspike 0.90 0.00
\sparkspike 1.00 0.10
\sparkbottomline
\end{sparkline}
\renewcommand{\sparklineheight}{1.75}}
&$
\begin{array}{c}
\scriptstyle{0.00} \\[-6pt]
\scriptscriptstyle{(0.000, 6801.472)}
\end{array}
$
\noindent\parbox[p]{4ex}{\renewcommand{\sparklineheight}{2.75}
\begin{sparkline}{4}
\definecolor{sparkspikecolor}{named}{red}
\sparkspike 0.10 0.70
\definecolor{sparkspikecolor}{named}{black}
\sparkspike 0.20 0.00
\sparkspike 0.30 0.00
\sparkspike 0.40 0.10
\sparkspike 0.50 0.10
\sparkspike 0.60 0.00
\sparkspike 0.70 0.00
\sparkspike 0.80 0.00
\sparkspike 0.90 0.00
\sparkspike 1.00 0.10
\sparkbottomline
\end{sparkline}
\renewcommand{\sparklineheight}{1.75}}
&$
\begin{array}{c}
\scriptstyle{9.65039} \\[-6pt]
\scriptscriptstyle{\pm0.150536}
\end{array}
$
\noindent\parbox[p]{4ex}{\renewcommand{\sparklineheight}{2.75}
\begin{sparkline}{4}
\sparkspike 0.10 0.10
\sparkspike 0.20 0.10
\sparkspike 0.30 0.20
\definecolor{sparkspikecolor}{named}{red}
\sparkspike 0.40 0.20
\definecolor{sparkspikecolor}{named}{black}
\sparkspike 0.50 0.10
\sparkspike 0.60 0.10
\sparkspike 0.70 0.00
\sparkspike 0.80 0.00
\sparkspike 0.90 0.00
\sparkspike 1.00 0.20
\sparkbottomline
\end{sparkline}
\renewcommand{\sparklineheight}{1.75}}
\\ 
rx-scrabble&\begin{minipage}[c][\blankheight]{0pt}\end{minipage}&&\multicolumn{1}{l}{\badinconsistent \scriptsize($7$\slowdown, $3$\warmup)}&$
\begin{array}{c}
\scriptstyle{76.0} \\[-6pt]
\scriptscriptstyle{(37.5, 121.3)}
\end{array}
$
\noindent\parbox[p]{4ex}{\renewcommand{\sparklineheight}{2.75}
\begin{sparkline}{4}
\sparkspike 0.10 0.10
\sparkspike 0.20 0.00
\sparkspike 0.30 0.00
\sparkspike 0.40 0.00
\sparkspike 0.50 0.00
\definecolor{sparkspikecolor}{named}{red}
\sparkspike 0.60 0.60
\definecolor{sparkspikecolor}{named}{black}
\sparkspike 0.70 0.20
\sparkspike 0.80 0.00
\sparkspike 0.90 0.00
\sparkspike 1.00 0.10
\sparkbottomline
\end{sparkline}
\renewcommand{\sparklineheight}{1.75}}
&$
\begin{array}{c}
\scriptstyle{17.30} \\[-6pt]
\scriptscriptstyle{(8.678, 27.051)}
\end{array}
$
\noindent\parbox[p]{4ex}{\renewcommand{\sparklineheight}{2.75}
\begin{sparkline}{4}
\sparkspike 0.10 0.10
\sparkspike 0.20 0.00
\sparkspike 0.30 0.00
\sparkspike 0.40 0.00
\sparkspike 0.50 0.10
\definecolor{sparkspikecolor}{named}{red}
\sparkspike 0.60 0.50
\definecolor{sparkspikecolor}{named}{black}
\sparkspike 0.70 0.20
\sparkspike 0.80 0.00
\sparkspike 0.90 0.00
\sparkspike 1.00 0.10
\sparkbottomline
\end{sparkline}
\renewcommand{\sparklineheight}{1.75}}
&$
\begin{array}{c}
\scriptstyle{0.21581} \\[-6pt]
\scriptscriptstyle{\pm0.004181}
\end{array}
$
\noindent\parbox[p]{4ex}{\renewcommand{\sparklineheight}{2.75}
\begin{sparkline}{4}
\sparkspike 0.10 0.20
\sparkspike 0.20 0.10
\sparkspike 0.30 0.00
\definecolor{sparkspikecolor}{named}{red}
\sparkspike 0.40 0.30
\definecolor{sparkspikecolor}{named}{black}
\sparkspike 0.50 0.00
\sparkspike 0.60 0.20
\sparkspike 0.70 0.00
\sparkspike 0.80 0.00
\sparkspike 0.90 0.00
\sparkspike 1.00 0.20
\sparkbottomline
\end{sparkline}
\renewcommand{\sparklineheight}{1.75}}
\\ 
scala-kmeans&\begin{minipage}[c][\blankheight]{0pt}\end{minipage}&&\multicolumn{1}{l}{\badinconsistent \scriptsize($7$\warmup, $2$\nosteadystate, $1$\slowdown)}&\begin{minipage}[c][\blankheight]{0pt}\end{minipage}&\begin{minipage}[c][\blankheight]{0pt}\end{minipage}&\begin{minipage}[c][\blankheight]{0pt}\end{minipage}\\ 
scala-stm-bench7&\begin{minipage}[c][\blankheight]{0pt}\end{minipage}&&\multicolumn{1}{l}{\warmup}&$
\begin{array}{c}
\scriptstyle{88.0} \\[-6pt]
\scriptscriptstyle{(49.1, 185.6)}
\end{array}
$
\noindent\parbox[p]{4ex}{\renewcommand{\sparklineheight}{2.75}
\begin{sparkline}{4}
\sparkspike 0.10 0.10
\definecolor{sparkspikecolor}{named}{red}
\sparkspike 0.20 0.40
\definecolor{sparkspikecolor}{named}{black}
\sparkspike 0.30 0.00
\sparkspike 0.40 0.10
\sparkspike 0.50 0.00
\sparkspike 0.60 0.20
\sparkspike 0.70 0.10
\sparkspike 0.80 0.00
\sparkspike 0.90 0.00
\sparkspike 1.00 0.10
\sparkbottomline
\end{sparkline}
\renewcommand{\sparklineheight}{1.75}}
&$
\begin{array}{c}
\scriptstyle{76.13} \\[-6pt]
\scriptscriptstyle{(43.389, 156.869)}
\end{array}
$
\noindent\parbox[p]{4ex}{\renewcommand{\sparklineheight}{2.75}
\begin{sparkline}{4}
\sparkspike 0.10 0.10
\definecolor{sparkspikecolor}{named}{red}
\sparkspike 0.20 0.40
\definecolor{sparkspikecolor}{named}{black}
\sparkspike 0.30 0.00
\sparkspike 0.40 0.10
\sparkspike 0.50 0.00
\sparkspike 0.60 0.20
\sparkspike 0.70 0.10
\sparkspike 0.80 0.00
\sparkspike 0.90 0.00
\sparkspike 1.00 0.10
\sparkbottomline
\end{sparkline}
\renewcommand{\sparklineheight}{1.75}}
&$
\begin{array}{c}
\scriptstyle{0.81315} \\[-6pt]
\scriptscriptstyle{\pm0.013074}
\end{array}
$
\noindent\parbox[p]{4ex}{\renewcommand{\sparklineheight}{2.75}
\begin{sparkline}{4}
\sparkspike 0.10 0.10
\sparkspike 0.20 0.10
\sparkspike 0.30 0.00
\definecolor{sparkspikecolor}{named}{red}
\sparkspike 0.40 0.30
\definecolor{sparkspikecolor}{named}{black}
\sparkspike 0.50 0.10
\sparkspike 0.60 0.10
\sparkspike 0.70 0.00
\sparkspike 0.80 0.10
\sparkspike 0.90 0.00
\sparkspike 1.00 0.20
\sparkbottomline
\end{sparkline}
\renewcommand{\sparklineheight}{1.75}}
\\ 
scrabble&\begin{minipage}[c][\blankheight]{0pt}\end{minipage}&&\multicolumn{1}{l}{\goodinconsistent \scriptsize($8$\flatc, $2$\warmup)}&$
\begin{array}{c}
\scriptstyle{1.0} \\[-6pt]
\scriptscriptstyle{(1.0, 11.0)}
\end{array}
$
\noindent\parbox[p]{4ex}{\renewcommand{\sparklineheight}{2.75}
\begin{sparkline}{4}
\definecolor{sparkspikecolor}{named}{red}
\sparkspike 0.10 0.80
\definecolor{sparkspikecolor}{named}{black}
\sparkspike 0.20 0.00
\sparkspike 0.30 0.00
\sparkspike 0.40 0.00
\sparkspike 0.50 0.00
\sparkspike 0.60 0.00
\sparkspike 0.70 0.00
\sparkspike 0.80 0.00
\sparkspike 0.90 0.00
\sparkspike 1.00 0.20
\sparkbottomline
\end{sparkline}
\renewcommand{\sparklineheight}{1.75}}
&$
\begin{array}{c}
\scriptstyle{0.00} \\[-6pt]
\scriptscriptstyle{(0.000, 5.032)}
\end{array}
$
\noindent\parbox[p]{4ex}{\renewcommand{\sparklineheight}{2.75}
\begin{sparkline}{4}
\definecolor{sparkspikecolor}{named}{red}
\sparkspike 0.10 0.80
\definecolor{sparkspikecolor}{named}{black}
\sparkspike 0.20 0.00
\sparkspike 0.30 0.00
\sparkspike 0.40 0.00
\sparkspike 0.50 0.00
\sparkspike 0.60 0.00
\sparkspike 0.70 0.00
\sparkspike 0.80 0.00
\sparkspike 0.90 0.00
\sparkspike 1.00 0.20
\sparkbottomline
\end{sparkline}
\renewcommand{\sparklineheight}{1.75}}
&$
\begin{array}{c}
\scriptstyle{0.31257} \\[-6pt]
\scriptscriptstyle{\pm0.016842}
\end{array}
$
\noindent\parbox[p]{4ex}{\renewcommand{\sparklineheight}{2.75}
\begin{sparkline}{4}
\sparkspike 0.10 0.10
\sparkspike 0.20 0.10
\sparkspike 0.30 0.00
\sparkspike 0.40 0.00
\sparkspike 0.50 0.00
\sparkspike 0.60 0.00
\definecolor{sparkspikecolor}{named}{red}
\sparkspike 0.70 0.30
\definecolor{sparkspikecolor}{named}{black}
\sparkspike 0.80 0.10
\sparkspike 0.90 0.20
\sparkspike 1.00 0.20
\sparkbottomline
\end{sparkline}
\renewcommand{\sparklineheight}{1.75}}
\\ 
\hline
akka-uct&\begin{minipage}[c][\blankheight]{0pt}\end{minipage}&\multirow{20}{*}{\rotatebox[origin=c]{90}{graal-ce-hotspot}}&\multicolumn{1}{l}{\badinconsistent \scriptsize($6$\flatc, $3$\warmup, $1$\nosteadystate)}&\begin{minipage}[c][\blankheight]{0pt}\end{minipage}&\begin{minipage}[c][\blankheight]{0pt}\end{minipage}&\begin{minipage}[c][\blankheight]{0pt}\end{minipage}\\ 
als&\begin{minipage}[c][\blankheight]{0pt}\end{minipage}&&\multicolumn{1}{l}{\goodinconsistent \scriptsize($8$\flatc, $2$\warmup)}&$
\begin{array}{c}
\scriptstyle{1.0} \\[-6pt]
\scriptscriptstyle{(1.0, 153.1)}
\end{array}
$
\noindent\parbox[p]{4ex}{\renewcommand{\sparklineheight}{2.75}
\begin{sparkline}{4}
\definecolor{sparkspikecolor}{named}{red}
\sparkspike 0.10 0.80
\definecolor{sparkspikecolor}{named}{black}
\sparkspike 0.20 0.00
\sparkspike 0.30 0.00
\sparkspike 0.40 0.00
\sparkspike 0.50 0.00
\sparkspike 0.60 0.00
\sparkspike 0.70 0.10
\sparkspike 0.80 0.00
\sparkspike 0.90 0.00
\sparkspike 1.00 0.10
\sparkbottomline
\end{sparkline}
\renewcommand{\sparklineheight}{1.75}}
&$
\begin{array}{c}
\scriptstyle{0.00} \\[-6pt]
\scriptscriptstyle{(0.000, 340.133)}
\end{array}
$
\noindent\parbox[p]{4ex}{\renewcommand{\sparklineheight}{2.75}
\begin{sparkline}{4}
\definecolor{sparkspikecolor}{named}{red}
\sparkspike 0.10 0.80
\definecolor{sparkspikecolor}{named}{black}
\sparkspike 0.20 0.00
\sparkspike 0.30 0.00
\sparkspike 0.40 0.00
\sparkspike 0.50 0.00
\sparkspike 0.60 0.00
\sparkspike 0.70 0.10
\sparkspike 0.80 0.00
\sparkspike 0.90 0.00
\sparkspike 1.00 0.10
\sparkbottomline
\end{sparkline}
\renewcommand{\sparklineheight}{1.75}}
&$
\begin{array}{c}
\scriptstyle{2.13666} \\[-6pt]
\scriptscriptstyle{\pm0.020275}
\end{array}
$
\noindent\parbox[p]{4ex}{\renewcommand{\sparklineheight}{2.75}
\begin{sparkline}{4}
\sparkspike 0.10 0.10
\sparkspike 0.20 0.10
\sparkspike 0.30 0.00
\sparkspike 0.40 0.00
\sparkspike 0.50 0.00
\sparkspike 0.60 0.00
\sparkspike 0.70 0.00
\sparkspike 0.80 0.00
\definecolor{sparkspikecolor}{named}{red}
\sparkspike 0.90 0.50
\definecolor{sparkspikecolor}{named}{black}
\sparkspike 1.00 0.30
\sparkbottomline
\end{sparkline}
\renewcommand{\sparklineheight}{1.75}}
\\ 
chi-square&\begin{minipage}[c][\blankheight]{0pt}\end{minipage}&&\multicolumn{1}{l}{\badinconsistent \scriptsize($4$\flatc, $4$\warmup, $1$\nosteadystate, $1$\slowdown)}&\begin{minipage}[c][\blankheight]{0pt}\end{minipage}&\begin{minipage}[c][\blankheight]{0pt}\end{minipage}&\begin{minipage}[c][\blankheight]{0pt}\end{minipage}\\ 
db-shootout&\begin{minipage}[c][\blankheight]{0pt}\end{minipage}&&\multicolumn{1}{l}{\flatc}&\begin{minipage}[c][\blankheight]{0pt}\end{minipage}&\begin{minipage}[c][\blankheight]{0pt}\end{minipage}&$
\begin{array}{c}
\scriptstyle{5.62052} \\[-6pt]
\scriptscriptstyle{\pm0.081742}
\end{array}
$
\noindent\parbox[p]{4ex}{\renewcommand{\sparklineheight}{2.75}
\begin{sparkline}{4}
\sparkspike 0.10 0.20
\sparkspike 0.20 0.00
\sparkspike 0.30 0.10
\sparkspike 0.40 0.10
\definecolor{sparkspikecolor}{named}{red}
\sparkspike 0.50 0.20
\definecolor{sparkspikecolor}{named}{black}
\sparkspike 0.60 0.00
\sparkspike 0.70 0.30
\sparkspike 0.80 0.00
\sparkspike 0.90 0.00
\sparkspike 1.00 0.10
\sparkbottomline
\end{sparkline}
\renewcommand{\sparklineheight}{1.75}}
\\ 
dec-tree&\begin{minipage}[c][\blankheight]{0pt}\end{minipage}&&\multicolumn{1}{l}{\badinconsistent \scriptsize($6$\nosteadystate, $4$\warmup)}&\begin{minipage}[c][\blankheight]{0pt}\end{minipage}&\begin{minipage}[c][\blankheight]{0pt}\end{minipage}&\begin{minipage}[c][\blankheight]{0pt}\end{minipage}\\ 
dotty&\begin{minipage}[c][\blankheight]{0pt}\end{minipage}&&\multicolumn{1}{l}{\warmup}&$
\begin{array}{c}
\scriptstyle{112.5} \\[-6pt]
\scriptscriptstyle{(101.2, 125.5)}
\end{array}
$
\noindent\parbox[p]{4ex}{\renewcommand{\sparklineheight}{2.75}
\begin{sparkline}{4}
\sparkspike 0.10 0.10
\sparkspike 0.20 0.20
\sparkspike 0.30 0.00
\definecolor{sparkspikecolor}{named}{red}
\sparkspike 0.40 0.20
\definecolor{sparkspikecolor}{named}{black}
\sparkspike 0.50 0.00
\sparkspike 0.60 0.30
\sparkspike 0.70 0.10
\sparkspike 0.80 0.00
\sparkspike 0.90 0.00
\sparkspike 1.00 0.10
\sparkbottomline
\end{sparkline}
\renewcommand{\sparklineheight}{1.75}}
&$
\begin{array}{c}
\scriptstyle{171.91} \\[-6pt]
\scriptscriptstyle{(156.735, 190.970)}
\end{array}
$
\noindent\parbox[p]{4ex}{\renewcommand{\sparklineheight}{2.75}
\begin{sparkline}{4}
\sparkspike 0.10 0.10
\sparkspike 0.20 0.10
\sparkspike 0.30 0.20
\definecolor{sparkspikecolor}{named}{red}
\sparkspike 0.40 0.10
\definecolor{sparkspikecolor}{named}{black}
\sparkspike 0.50 0.20
\sparkspike 0.60 0.10
\sparkspike 0.70 0.10
\sparkspike 0.80 0.00
\sparkspike 0.90 0.00
\sparkspike 1.00 0.10
\sparkbottomline
\end{sparkline}
\renewcommand{\sparklineheight}{1.75}}
&$
\begin{array}{c}
\scriptstyle{1.36074} \\[-6pt]
\scriptscriptstyle{\pm0.027632}
\end{array}
$
\noindent\parbox[p]{4ex}{\renewcommand{\sparklineheight}{2.75}
\begin{sparkline}{4}
\sparkspike 0.10 0.10
\sparkspike 0.20 0.00
\definecolor{sparkspikecolor}{named}{red}
\sparkspike 0.30 0.40
\definecolor{sparkspikecolor}{named}{black}
\sparkspike 0.40 0.10
\sparkspike 0.50 0.10
\sparkspike 0.60 0.10
\sparkspike 0.70 0.00
\sparkspike 0.80 0.10
\sparkspike 0.90 0.00
\sparkspike 1.00 0.10
\sparkbottomline
\end{sparkline}
\renewcommand{\sparklineheight}{1.75}}
\\ 
fj-kmeans&\begin{minipage}[c][\blankheight]{0pt}\end{minipage}&&\multicolumn{1}{l}{\goodinconsistent \scriptsize($6$\warmup, $4$\flatc)}&$
\begin{array}{c}
\scriptstyle{2.0} \\[-6pt]
\scriptscriptstyle{(1.0, 3.1)}
\end{array}
$
\noindent\parbox[p]{4ex}{\renewcommand{\sparklineheight}{2.75}
\begin{sparkline}{4}
\sparkspike 0.10 0.40
\sparkspike 0.20 0.00
\sparkspike 0.30 0.00
\definecolor{sparkspikecolor}{named}{red}
\sparkspike 0.40 0.50
\definecolor{sparkspikecolor}{named}{black}
\sparkspike 0.50 0.00
\sparkspike 0.60 0.00
\sparkspike 0.70 0.00
\sparkspike 0.80 0.00
\sparkspike 0.90 0.00
\sparkspike 1.00 0.10
\sparkbottomline
\end{sparkline}
\renewcommand{\sparklineheight}{1.75}}
&$
\begin{array}{c}
\scriptstyle{4.96} \\[-6pt]
\scriptscriptstyle{(0.000, 9.890)}
\end{array}
$
\noindent\parbox[p]{4ex}{\renewcommand{\sparklineheight}{2.75}
\begin{sparkline}{4}
\sparkspike 0.10 0.40
\sparkspike 0.20 0.00
\sparkspike 0.30 0.00
\definecolor{sparkspikecolor}{named}{red}
\sparkspike 0.40 0.50
\definecolor{sparkspikecolor}{named}{black}
\sparkspike 0.50 0.00
\sparkspike 0.60 0.00
\sparkspike 0.70 0.00
\sparkspike 0.80 0.00
\sparkspike 0.90 0.00
\sparkspike 1.00 0.10
\sparkbottomline
\end{sparkline}
\renewcommand{\sparklineheight}{1.75}}
&$
\begin{array}{c}
\scriptstyle{4.32436} \\[-6pt]
\scriptscriptstyle{\pm0.386333}
\end{array}
$
\noindent\parbox[p]{4ex}{\renewcommand{\sparklineheight}{2.75}
\begin{sparkline}{4}
\sparkspike 0.10 0.30
\definecolor{sparkspikecolor}{named}{red}
\sparkspike 0.20 0.20
\definecolor{sparkspikecolor}{named}{black}
\sparkspike 0.30 0.10
\sparkspike 0.40 0.10
\sparkspike 0.50 0.00
\sparkspike 0.60 0.10
\sparkspike 0.70 0.00
\sparkspike 0.80 0.10
\sparkspike 0.90 0.00
\sparkspike 1.00 0.10
\sparkbottomline
\end{sparkline}
\renewcommand{\sparklineheight}{1.75}}
\\ 
future-genetic&\begin{minipage}[c][\blankheight]{0pt}\end{minipage}&&\multicolumn{1}{l}{\badinconsistent \scriptsize($7$\warmup, $2$\nosteadystate, $1$\flatc)}&\begin{minipage}[c][\blankheight]{0pt}\end{minipage}&\begin{minipage}[c][\blankheight]{0pt}\end{minipage}&\begin{minipage}[c][\blankheight]{0pt}\end{minipage}\\ 
gauss-mix&\begin{minipage}[c][\blankheight]{0pt}\end{minipage}&&\multicolumn{1}{l}{\warmup}&$
\begin{array}{c}
\scriptstyle{85.5} \\[-6pt]
\scriptscriptstyle{(82.9, 136.5)}
\end{array}
$
\noindent\parbox[p]{4ex}{\renewcommand{\sparklineheight}{2.75}
\begin{sparkline}{4}
\definecolor{sparkspikecolor}{named}{red}
\sparkspike 0.10 0.70
\definecolor{sparkspikecolor}{named}{black}
\sparkspike 0.20 0.10
\sparkspike 0.30 0.10
\sparkspike 0.40 0.00
\sparkspike 0.50 0.00
\sparkspike 0.60 0.00
\sparkspike 0.70 0.00
\sparkspike 0.80 0.00
\sparkspike 0.90 0.00
\sparkspike 1.00 0.10
\sparkbottomline
\end{sparkline}
\renewcommand{\sparklineheight}{1.75}}
&$
\begin{array}{c}
\scriptstyle{76.30} \\[-6pt]
\scriptscriptstyle{(72.632, 118.745)}
\end{array}
$
\noindent\parbox[p]{4ex}{\renewcommand{\sparklineheight}{2.75}
\begin{sparkline}{4}
\definecolor{sparkspikecolor}{named}{red}
\sparkspike 0.10 0.60
\definecolor{sparkspikecolor}{named}{black}
\sparkspike 0.20 0.20
\sparkspike 0.30 0.10
\sparkspike 0.40 0.00
\sparkspike 0.50 0.00
\sparkspike 0.60 0.00
\sparkspike 0.70 0.00
\sparkspike 0.80 0.00
\sparkspike 0.90 0.00
\sparkspike 1.00 0.10
\sparkbottomline
\end{sparkline}
\renewcommand{\sparklineheight}{1.75}}
&$
\begin{array}{c}
\scriptstyle{0.83840} \\[-6pt]
\scriptscriptstyle{\pm0.016073}
\end{array}
$
\noindent\parbox[p]{4ex}{\renewcommand{\sparklineheight}{2.75}
\begin{sparkline}{4}
\sparkspike 0.10 0.10
\sparkspike 0.20 0.30
\definecolor{sparkspikecolor}{named}{red}
\sparkspike 0.30 0.20
\definecolor{sparkspikecolor}{named}{black}
\sparkspike 0.40 0.00
\sparkspike 0.50 0.20
\sparkspike 0.60 0.00
\sparkspike 0.70 0.00
\sparkspike 0.80 0.10
\sparkspike 0.90 0.00
\sparkspike 1.00 0.10
\sparkbottomline
\end{sparkline}
\renewcommand{\sparklineheight}{1.75}}
\\ 
log-regression&\begin{minipage}[c][\blankheight]{0pt}\end{minipage}&&\multicolumn{1}{l}{\nosteadystate}&\begin{minipage}[c][\blankheight]{0pt}\end{minipage}&\begin{minipage}[c][\blankheight]{0pt}\end{minipage}&\begin{minipage}[c][\blankheight]{0pt}\end{minipage}\\ 
mnemonics&\begin{minipage}[c][\blankheight]{0pt}\end{minipage}&&\multicolumn{1}{l}{\badinconsistent \scriptsize($8$\warmup, $2$\slowdown)}&$
\begin{array}{c}
\scriptstyle{211.0} \\[-6pt]
\scriptscriptstyle{(3.0, 542.3)}
\end{array}
$
\noindent\parbox[p]{4ex}{\renewcommand{\sparklineheight}{2.75}
\begin{sparkline}{4}
\definecolor{sparkspikecolor}{named}{red}
\sparkspike 0.10 0.50
\definecolor{sparkspikecolor}{named}{black}
\sparkspike 0.20 0.00
\sparkspike 0.30 0.00
\sparkspike 0.40 0.00
\sparkspike 0.50 0.00
\sparkspike 0.60 0.00
\sparkspike 0.70 0.10
\sparkspike 0.80 0.30
\sparkspike 0.90 0.00
\sparkspike 1.00 0.10
\sparkbottomline
\end{sparkline}
\renewcommand{\sparklineheight}{1.75}}
&$
\begin{array}{c}
\scriptstyle{680.23} \\[-6pt]
\scriptscriptstyle{(7.269, 1760.362)}
\end{array}
$
\noindent\parbox[p]{4ex}{\renewcommand{\sparklineheight}{2.75}
\begin{sparkline}{4}
\definecolor{sparkspikecolor}{named}{red}
\sparkspike 0.10 0.50
\definecolor{sparkspikecolor}{named}{black}
\sparkspike 0.20 0.00
\sparkspike 0.30 0.00
\sparkspike 0.40 0.00
\sparkspike 0.50 0.00
\sparkspike 0.60 0.00
\sparkspike 0.70 0.10
\sparkspike 0.80 0.30
\sparkspike 0.90 0.00
\sparkspike 1.00 0.10
\sparkbottomline
\end{sparkline}
\renewcommand{\sparklineheight}{1.75}}
&$
\begin{array}{c}
\scriptstyle{3.25706} \\[-6pt]
\scriptscriptstyle{\pm0.047435}
\end{array}
$
\noindent\parbox[p]{4ex}{\renewcommand{\sparklineheight}{2.75}
\begin{sparkline}{4}
\sparkspike 0.10 0.10
\sparkspike 0.20 0.00
\sparkspike 0.30 0.00
\sparkspike 0.40 0.00
\sparkspike 0.50 0.10
\sparkspike 0.60 0.20
\definecolor{sparkspikecolor}{named}{red}
\sparkspike 0.70 0.10
\definecolor{sparkspikecolor}{named}{black}
\sparkspike 0.80 0.20
\sparkspike 0.90 0.10
\sparkspike 1.00 0.20
\sparkbottomline
\end{sparkline}
\renewcommand{\sparklineheight}{1.75}}
\\ 
naive-bayes&\begin{minipage}[c][\blankheight]{0pt}\end{minipage}&&\multicolumn{1}{l}{\goodinconsistent \scriptsize($8$\warmup, $2$\flatc)}&$
\begin{array}{c}
\scriptstyle{46.5} \\[-6pt]
\scriptscriptstyle{(1.0, 143.3)}
\end{array}
$
\noindent\parbox[p]{4ex}{\renewcommand{\sparklineheight}{2.75}
\begin{sparkline}{4}
\sparkspike 0.10 0.20
\sparkspike 0.20 0.00
\definecolor{sparkspikecolor}{named}{red}
\sparkspike 0.30 0.30
\definecolor{sparkspikecolor}{named}{black}
\sparkspike 0.40 0.10
\sparkspike 0.50 0.20
\sparkspike 0.60 0.00
\sparkspike 0.70 0.00
\sparkspike 0.80 0.00
\sparkspike 0.90 0.10
\sparkspike 1.00 0.10
\sparkbottomline
\end{sparkline}
\renewcommand{\sparklineheight}{1.75}}
&$
\begin{array}{c}
\scriptstyle{112.47} \\[-6pt]
\scriptscriptstyle{(0.000, 329.625)}
\end{array}
$
\noindent\parbox[p]{4ex}{\renewcommand{\sparklineheight}{2.75}
\begin{sparkline}{4}
\sparkspike 0.10 0.20
\sparkspike 0.20 0.00
\definecolor{sparkspikecolor}{named}{red}
\sparkspike 0.30 0.30
\definecolor{sparkspikecolor}{named}{black}
\sparkspike 0.40 0.10
\sparkspike 0.50 0.10
\sparkspike 0.60 0.10
\sparkspike 0.70 0.00
\sparkspike 0.80 0.00
\sparkspike 0.90 0.10
\sparkspike 1.00 0.10
\sparkbottomline
\end{sparkline}
\renewcommand{\sparklineheight}{1.75}}
&$
\begin{array}{c}
\scriptstyle{2.23040} \\[-6pt]
\scriptscriptstyle{\pm0.025121}
\end{array}
$
\noindent\parbox[p]{4ex}{\renewcommand{\sparklineheight}{2.75}
\begin{sparkline}{4}
\definecolor{sparkspikecolor}{named}{red}
\sparkspike 0.10 0.70
\definecolor{sparkspikecolor}{named}{black}
\sparkspike 0.20 0.10
\sparkspike 0.30 0.00
\sparkspike 0.40 0.00
\sparkspike 0.50 0.00
\sparkspike 0.60 0.00
\sparkspike 0.70 0.00
\sparkspike 0.80 0.00
\sparkspike 0.90 0.00
\sparkspike 1.00 0.20
\sparkbottomline
\end{sparkline}
\renewcommand{\sparklineheight}{1.75}}
\\ 
neo4j-analytics&\begin{minipage}[c][\blankheight]{0pt}\end{minipage}&&\multicolumn{1}{l}{\badinconsistent \scriptsize($6$\warmup, $3$\nosteadystate, $1$\flatc)}&\begin{minipage}[c][\blankheight]{0pt}\end{minipage}&\begin{minipage}[c][\blankheight]{0pt}\end{minipage}&\begin{minipage}[c][\blankheight]{0pt}\end{minipage}\\ 
par-mnemonics&\begin{minipage}[c][\blankheight]{0pt}\end{minipage}&&\multicolumn{1}{l}{\warmup}&$
\begin{array}{c}
\scriptstyle{98.0} \\[-6pt]
\scriptscriptstyle{(5.0, 513.2)}
\end{array}
$
\noindent\parbox[p]{4ex}{\renewcommand{\sparklineheight}{2.75}
\begin{sparkline}{4}
\sparkspike 0.10 0.40
\definecolor{sparkspikecolor}{named}{red}
\sparkspike 0.20 0.30
\definecolor{sparkspikecolor}{named}{black}
\sparkspike 0.30 0.10
\sparkspike 0.40 0.00
\sparkspike 0.50 0.10
\sparkspike 0.60 0.00
\sparkspike 0.70 0.00
\sparkspike 0.80 0.00
\sparkspike 0.90 0.00
\sparkspike 1.00 0.10
\sparkbottomline
\end{sparkline}
\renewcommand{\sparklineheight}{1.75}}
&$
\begin{array}{c}
\scriptstyle{250.38} \\[-6pt]
\scriptscriptstyle{(11.020, 1293.787)}
\end{array}
$
\noindent\parbox[p]{4ex}{\renewcommand{\sparklineheight}{2.75}
\begin{sparkline}{4}
\sparkspike 0.10 0.40
\definecolor{sparkspikecolor}{named}{red}
\sparkspike 0.20 0.30
\definecolor{sparkspikecolor}{named}{black}
\sparkspike 0.30 0.10
\sparkspike 0.40 0.00
\sparkspike 0.50 0.10
\sparkspike 0.60 0.00
\sparkspike 0.70 0.00
\sparkspike 0.80 0.00
\sparkspike 0.90 0.00
\sparkspike 1.00 0.10
\sparkbottomline
\end{sparkline}
\renewcommand{\sparklineheight}{1.75}}
&$
\begin{array}{c}
\scriptstyle{2.57589} \\[-6pt]
\scriptscriptstyle{\pm0.146179}
\end{array}
$
\noindent\parbox[p]{4ex}{\renewcommand{\sparklineheight}{2.75}
\begin{sparkline}{4}
\sparkspike 0.10 0.20
\sparkspike 0.20 0.10
\sparkspike 0.30 0.00
\sparkspike 0.40 0.00
\sparkspike 0.50 0.00
\sparkspike 0.60 0.00
\sparkspike 0.70 0.00
\definecolor{sparkspikecolor}{named}{red}
\sparkspike 0.80 0.40
\definecolor{sparkspikecolor}{named}{black}
\sparkspike 0.90 0.00
\sparkspike 1.00 0.30
\sparkbottomline
\end{sparkline}
\renewcommand{\sparklineheight}{1.75}}
\\ 
philosophers&\begin{minipage}[c][\blankheight]{0pt}\end{minipage}&&\multicolumn{1}{l}{\badinconsistent \scriptsize($9$\warmup, $1$\slowdown)}&$
\begin{array}{c}
\scriptstyle{95.5} \\[-6pt]
\scriptscriptstyle{(4.7, 897.5)}
\end{array}
$
\noindent\parbox[p]{4ex}{\renewcommand{\sparklineheight}{2.75}
\begin{sparkline}{4}
\definecolor{sparkspikecolor}{named}{red}
\sparkspike 0.10 0.50
\definecolor{sparkspikecolor}{named}{black}
\sparkspike 0.20 0.20
\sparkspike 0.30 0.20
\sparkspike 0.40 0.00
\sparkspike 0.50 0.00
\sparkspike 0.60 0.00
\sparkspike 0.70 0.00
\sparkspike 0.80 0.00
\sparkspike 0.90 0.00
\sparkspike 1.00 0.10
\sparkbottomline
\end{sparkline}
\renewcommand{\sparklineheight}{1.75}}
&$
\begin{array}{c}
\scriptstyle{129.23} \\[-6pt]
\scriptscriptstyle{(5.126, 1145.639)}
\end{array}
$
\noindent\parbox[p]{4ex}{\renewcommand{\sparklineheight}{2.75}
\begin{sparkline}{4}
\definecolor{sparkspikecolor}{named}{red}
\sparkspike 0.10 0.50
\definecolor{sparkspikecolor}{named}{black}
\sparkspike 0.20 0.20
\sparkspike 0.30 0.20
\sparkspike 0.40 0.00
\sparkspike 0.50 0.00
\sparkspike 0.60 0.00
\sparkspike 0.70 0.00
\sparkspike 0.80 0.00
\sparkspike 0.90 0.00
\sparkspike 1.00 0.10
\sparkbottomline
\end{sparkline}
\renewcommand{\sparklineheight}{1.75}}
&$
\begin{array}{c}
\scriptstyle{1.22564} \\[-6pt]
\scriptscriptstyle{\pm0.085480}
\end{array}
$
\noindent\parbox[p]{4ex}{\renewcommand{\sparklineheight}{2.75}
\begin{sparkline}{4}
\sparkspike 0.10 0.20
\sparkspike 0.20 0.00
\sparkspike 0.30 0.00
\sparkspike 0.40 0.10
\sparkspike 0.50 0.10
\definecolor{sparkspikecolor}{named}{red}
\sparkspike 0.60 0.40
\definecolor{sparkspikecolor}{named}{black}
\sparkspike 0.70 0.00
\sparkspike 0.80 0.00
\sparkspike 0.90 0.00
\sparkspike 1.00 0.20
\sparkbottomline
\end{sparkline}
\renewcommand{\sparklineheight}{1.75}}
\\ 
reactors&\begin{minipage}[c][\blankheight]{0pt}\end{minipage}&&\multicolumn{1}{l}{\badinconsistent \scriptsize($6$\slowdown, $4$\flatc)}&$
\begin{array}{c}
\scriptstyle{328.5} \\[-6pt]
\scriptscriptstyle{(1.0, 812.9)}
\end{array}
$
\noindent\parbox[p]{4ex}{\renewcommand{\sparklineheight}{2.75}
\begin{sparkline}{4}
\sparkspike 0.10 0.40
\sparkspike 0.20 0.00
\sparkspike 0.30 0.00
\definecolor{sparkspikecolor}{named}{red}
\sparkspike 0.40 0.20
\definecolor{sparkspikecolor}{named}{black}
\sparkspike 0.50 0.10
\sparkspike 0.60 0.00
\sparkspike 0.70 0.10
\sparkspike 0.80 0.00
\sparkspike 0.90 0.00
\sparkspike 1.00 0.20
\sparkbottomline
\end{sparkline}
\renewcommand{\sparklineheight}{1.75}}
&$
\begin{array}{c}
\scriptstyle{3383.74} \\[-6pt]
\scriptscriptstyle{(0.000, 8481.953)}
\end{array}
$
\noindent\parbox[p]{4ex}{\renewcommand{\sparklineheight}{2.75}
\begin{sparkline}{4}
\sparkspike 0.10 0.40
\sparkspike 0.20 0.00
\sparkspike 0.30 0.00
\definecolor{sparkspikecolor}{named}{red}
\sparkspike 0.40 0.20
\definecolor{sparkspikecolor}{named}{black}
\sparkspike 0.50 0.10
\sparkspike 0.60 0.00
\sparkspike 0.70 0.10
\sparkspike 0.80 0.00
\sparkspike 0.90 0.00
\sparkspike 1.00 0.20
\sparkbottomline
\end{sparkline}
\renewcommand{\sparklineheight}{1.75}}
&$
\begin{array}{c}
\scriptstyle{10.49981} \\[-6pt]
\scriptscriptstyle{\pm0.203252}
\end{array}
$
\noindent\parbox[p]{4ex}{\renewcommand{\sparklineheight}{2.75}
\begin{sparkline}{4}
\sparkspike 0.10 0.10
\sparkspike 0.20 0.10
\sparkspike 0.30 0.10
\sparkspike 0.40 0.10
\definecolor{sparkspikecolor}{named}{red}
\sparkspike 0.50 0.20
\definecolor{sparkspikecolor}{named}{black}
\sparkspike 0.60 0.20
\sparkspike 0.70 0.00
\sparkspike 0.80 0.00
\sparkspike 0.90 0.00
\sparkspike 1.00 0.20
\sparkbottomline
\end{sparkline}
\renewcommand{\sparklineheight}{1.75}}
\\ 
rx-scrabble&\begin{minipage}[c][\blankheight]{0pt}\end{minipage}&&\multicolumn{1}{l}{\badinconsistent \scriptsize($8$\warmup, $2$\slowdown)}&$
\begin{array}{c}
\scriptstyle{77.0} \\[-6pt]
\scriptscriptstyle{(44.1, 94.0)}
\end{array}
$
\noindent\parbox[p]{4ex}{\renewcommand{\sparklineheight}{2.75}
\begin{sparkline}{4}
\sparkspike 0.10 0.10
\sparkspike 0.20 0.00
\sparkspike 0.30 0.00
\sparkspike 0.40 0.00
\sparkspike 0.50 0.00
\sparkspike 0.60 0.00
\sparkspike 0.70 0.00
\definecolor{sparkspikecolor}{named}{red}
\sparkspike 0.80 0.60
\definecolor{sparkspikecolor}{named}{black}
\sparkspike 0.90 0.00
\sparkspike 1.00 0.30
\sparkbottomline
\end{sparkline}
\renewcommand{\sparklineheight}{1.75}}
&$
\begin{array}{c}
\scriptstyle{16.82} \\[-6pt]
\scriptscriptstyle{(9.872, 20.362)}
\end{array}
$
\noindent\parbox[p]{4ex}{\renewcommand{\sparklineheight}{2.75}
\begin{sparkline}{4}
\sparkspike 0.10 0.10
\sparkspike 0.20 0.00
\sparkspike 0.30 0.00
\sparkspike 0.40 0.00
\sparkspike 0.50 0.00
\sparkspike 0.60 0.00
\sparkspike 0.70 0.00
\definecolor{sparkspikecolor}{named}{red}
\sparkspike 0.80 0.50
\definecolor{sparkspikecolor}{named}{black}
\sparkspike 0.90 0.10
\sparkspike 1.00 0.30
\sparkbottomline
\end{sparkline}
\renewcommand{\sparklineheight}{1.75}}
&$
\begin{array}{c}
\scriptstyle{0.21248} \\[-6pt]
\scriptscriptstyle{\pm0.004011}
\end{array}
$
\noindent\parbox[p]{4ex}{\renewcommand{\sparklineheight}{2.75}
\begin{sparkline}{4}
\sparkspike 0.10 0.20
\sparkspike 0.20 0.10
\definecolor{sparkspikecolor}{named}{red}
\sparkspike 0.30 0.20
\definecolor{sparkspikecolor}{named}{black}
\sparkspike 0.40 0.00
\sparkspike 0.50 0.10
\sparkspike 0.60 0.20
\sparkspike 0.70 0.10
\sparkspike 0.80 0.00
\sparkspike 0.90 0.00
\sparkspike 1.00 0.10
\sparkbottomline
\end{sparkline}
\renewcommand{\sparklineheight}{1.75}}
\\ 
scala-kmeans&\begin{minipage}[c][\blankheight]{0pt}\end{minipage}&&\multicolumn{1}{l}{\badinconsistent \scriptsize($5$\nosteadystate, $5$\slowdown)}&\begin{minipage}[c][\blankheight]{0pt}\end{minipage}&\begin{minipage}[c][\blankheight]{0pt}\end{minipage}&\begin{minipage}[c][\blankheight]{0pt}\end{minipage}\\ 
scala-stm-bench7&\begin{minipage}[c][\blankheight]{0pt}\end{minipage}&&\multicolumn{1}{l}{\badinconsistent \scriptsize($9$\warmup, $1$\slowdown)}&$
\begin{array}{c}
\scriptstyle{437.5} \\[-6pt]
\scriptscriptstyle{(177.2, 1143.5)}
\end{array}
$
\noindent\parbox[p]{4ex}{\renewcommand{\sparklineheight}{2.75}
\begin{sparkline}{4}
\sparkspike 0.10 0.10
\sparkspike 0.20 0.30
\definecolor{sparkspikecolor}{named}{red}
\sparkspike 0.30 0.10
\definecolor{sparkspikecolor}{named}{black}
\sparkspike 0.40 0.00
\sparkspike 0.50 0.20
\sparkspike 0.60 0.00
\sparkspike 0.70 0.10
\sparkspike 0.80 0.00
\sparkspike 0.90 0.00
\sparkspike 1.00 0.20
\sparkbottomline
\end{sparkline}
\renewcommand{\sparklineheight}{1.75}}
&$
\begin{array}{c}
\scriptstyle{399.55} \\[-6pt]
\scriptscriptstyle{(157.817, 989.799)}
\end{array}
$
\noindent\parbox[p]{4ex}{\renewcommand{\sparklineheight}{2.75}
\begin{sparkline}{4}
\sparkspike 0.10 0.10
\sparkspike 0.20 0.30
\definecolor{sparkspikecolor}{named}{red}
\sparkspike 0.30 0.10
\definecolor{sparkspikecolor}{named}{black}
\sparkspike 0.40 0.00
\sparkspike 0.50 0.20
\sparkspike 0.60 0.00
\sparkspike 0.70 0.10
\sparkspike 0.80 0.00
\sparkspike 0.90 0.00
\sparkspike 1.00 0.20
\sparkbottomline
\end{sparkline}
\renewcommand{\sparklineheight}{1.75}}
&$
\begin{array}{c}
\scriptstyle{0.85001} \\[-6pt]
\scriptscriptstyle{\pm0.059300}
\end{array}
$
\noindent\parbox[p]{4ex}{\renewcommand{\sparklineheight}{2.75}
\begin{sparkline}{4}
\sparkspike 0.10 0.20
\definecolor{sparkspikecolor}{named}{red}
\sparkspike 0.20 0.30
\definecolor{sparkspikecolor}{named}{black}
\sparkspike 0.30 0.40
\sparkspike 0.40 0.00
\sparkspike 0.50 0.00
\sparkspike 0.60 0.00
\sparkspike 0.70 0.00
\sparkspike 0.80 0.00
\sparkspike 0.90 0.00
\sparkspike 1.00 0.10
\sparkbottomline
\end{sparkline}
\renewcommand{\sparklineheight}{1.75}}
\\ 
scrabble&\begin{minipage}[c][\blankheight]{0pt}\end{minipage}&&\multicolumn{1}{l}{\warmup}&$
\begin{array}{c}
\scriptstyle{12.0} \\[-6pt]
\scriptscriptstyle{(11.0, 12.0)}
\end{array}
$
\noindent\parbox[p]{4ex}{\renewcommand{\sparklineheight}{2.75}
\begin{sparkline}{4}
\sparkspike 0.10 0.20
\sparkspike 0.20 0.00
\sparkspike 0.30 0.00
\sparkspike 0.40 0.00
\sparkspike 0.50 0.00
\sparkspike 0.60 0.00
\sparkspike 0.70 0.00
\sparkspike 0.80 0.00
\sparkspike 0.90 0.00
\definecolor{sparkspikecolor}{named}{red}
\sparkspike 1.00 0.80
\definecolor{sparkspikecolor}{named}{black}
\sparkbottomline
\end{sparkline}
\renewcommand{\sparklineheight}{1.75}}
&$
\begin{array}{c}
\scriptstyle{5.08} \\[-6pt]
\scriptscriptstyle{(4.598, 5.193)}
\end{array}
$
\noindent\parbox[p]{4ex}{\renewcommand{\sparklineheight}{2.75}
\begin{sparkline}{4}
\sparkspike 0.10 0.10
\sparkspike 0.20 0.00
\sparkspike 0.30 0.00
\sparkspike 0.40 0.10
\sparkspike 0.50 0.00
\sparkspike 0.60 0.00
\sparkspike 0.70 0.00
\sparkspike 0.80 0.20
\definecolor{sparkspikecolor}{named}{red}
\sparkspike 0.90 0.30
\definecolor{sparkspikecolor}{named}{black}
\sparkspike 1.00 0.30
\sparkbottomline
\end{sparkline}
\renewcommand{\sparklineheight}{1.75}}
&$
\begin{array}{c}
\scriptstyle{0.36083} \\[-6pt]
\scriptscriptstyle{\pm0.006925}
\end{array}
$
\noindent\parbox[p]{4ex}{\renewcommand{\sparklineheight}{2.75}
\begin{sparkline}{4}
\sparkspike 0.10 0.10
\sparkspike 0.20 0.00
\sparkspike 0.30 0.00
\sparkspike 0.40 0.00
\sparkspike 0.50 0.20
\definecolor{sparkspikecolor}{named}{red}
\sparkspike 0.60 0.20
\definecolor{sparkspikecolor}{named}{black}
\sparkspike 0.70 0.00
\sparkspike 0.80 0.10
\sparkspike 0.90 0.20
\sparkspike 1.00 0.20
\sparkbottomline
\end{sparkline}
\renewcommand{\sparklineheight}{1.75}}
\\ 

\hline
\end{longtable}
}

{
\setlength\sparkspikewidth{1.5pt}
\definecolor{sparkbottomlinecolor}{gray}{0.4}
% Older versions of sparklines do not expose bottomlinethickness
\renewcommand{\sparkbottomline}[1][1]{\pgfsetlinewidth{0.2pt}%
  \color{sparkbottomlinecolor}%
  \pgfline{\pgfxy(0,0)}{\pgfxy(#1,0)}\color{sparklinecolor}}

\begin{longtable}{ll@{\hspace{0cm}}ll@{\hspace{-1cm}}r@{\hspace{0cm}}r@{\hspace{0cm}}r@{\hspace{0cm}}l@{\hspace{.3cm}}ll@{\hspace{-1cm}}r@{\hspace{0cm}}r@{\hspace{0cm}}r}
\multicolumn{1}{c}{\multirow{2}{*}{}}&&&\multicolumn{1}{c}{} &\multicolumn{1}{c}{Steady}&\multicolumn{1}{c}{Steady}&\multicolumn{1}{c}{Steady}\\&&&\multicolumn{1}{c}{Class.} &\multicolumn{1}{c}{iter (\#)} &\multicolumn{1}{c}{iter (s)}&\multicolumn{1}{c}{perf (s)} \\\hline
\endhead
akka-uct&\begin{minipage}[c][\blankheight]{0pt}\end{minipage}&\multirow{20}{*}{\rotatebox[origin=c]{90}{graal-ce}}&\multicolumn{1}{l}{\badinconsistent \scriptsize($5$\warmup, $4$\flatc, $1$\nosteadystate)}&\begin{minipage}[c][\blankheight]{0pt}\end{minipage}&\begin{minipage}[c][\blankheight]{0pt}\end{minipage}&\begin{minipage}[c][\blankheight]{0pt}\end{minipage}\\ 
als&\begin{minipage}[c][\blankheight]{0pt}\end{minipage}&&\multicolumn{1}{l}{\badinconsistent \scriptsize($5$\nosteadystate, $3$\flatc, $2$\warmup)}&\begin{minipage}[c][\blankheight]{0pt}\end{minipage}&\begin{minipage}[c][\blankheight]{0pt}\end{minipage}&\begin{minipage}[c][\blankheight]{0pt}\end{minipage}\\ 
chi-square&\begin{minipage}[c][\blankheight]{0pt}\end{minipage}&&\multicolumn{1}{l}{\goodinconsistent \scriptsize($9$\warmup, $1$\flatc)}&$
\begin{array}{c}
\scriptstyle{44.0} \\[-6pt]
\scriptscriptstyle{(19.9, 75.7)}
\end{array}
$
\noindent\parbox[p]{4ex}{\renewcommand{\sparklineheight}{2.75}
\begin{sparkline}{4}
\sparkspike 0.10 0.10
\sparkspike 0.20 0.00
\sparkspike 0.30 0.00
\sparkspike 0.40 0.00
\definecolor{sparkspikecolor}{named}{red}
\sparkspike 0.50 0.80
\definecolor{sparkspikecolor}{named}{black}
\sparkspike 0.60 0.00
\sparkspike 0.70 0.00
\sparkspike 0.80 0.00
\sparkspike 0.90 0.00
\sparkspike 1.00 0.10
\sparkbottomline
\end{sparkline}
\renewcommand{\sparklineheight}{1.75}}
&$
\begin{array}{c}
\scriptstyle{36.98} \\[-6pt]
\scriptscriptstyle{(16.218, 63.543)}
\end{array}
$
\noindent\parbox[p]{4ex}{\renewcommand{\sparklineheight}{2.75}
\begin{sparkline}{4}
\sparkspike 0.10 0.10
\sparkspike 0.20 0.00
\sparkspike 0.30 0.00
\sparkspike 0.40 0.00
\definecolor{sparkspikecolor}{named}{red}
\sparkspike 0.50 0.80
\definecolor{sparkspikecolor}{named}{black}
\sparkspike 0.60 0.00
\sparkspike 0.70 0.00
\sparkspike 0.80 0.00
\sparkspike 0.90 0.00
\sparkspike 1.00 0.10
\sparkbottomline
\end{sparkline}
\renewcommand{\sparklineheight}{1.75}}
&$
\begin{array}{c}
\scriptstyle{0.80429} \\[-6pt]
\scriptscriptstyle{\pm0.052808}
\end{array}
$
\noindent\parbox[p]{4ex}{\renewcommand{\sparklineheight}{2.75}
\begin{sparkline}{4}
\sparkspike 0.10 0.20
\sparkspike 0.20 0.20
\definecolor{sparkspikecolor}{named}{red}
\sparkspike 0.30 0.10
\definecolor{sparkspikecolor}{named}{black}
\sparkspike 0.40 0.10
\sparkspike 0.50 0.00
\sparkspike 0.60 0.00
\sparkspike 0.70 0.20
\sparkspike 0.80 0.00
\sparkspike 0.90 0.00
\sparkspike 1.00 0.20
\sparkbottomline
\end{sparkline}
\renewcommand{\sparklineheight}{1.75}}
\\ 
db-shootout&\begin{minipage}[c][\blankheight]{0pt}\end{minipage}&&\multicolumn{1}{l}{\flatc}&\begin{minipage}[c][\blankheight]{0pt}\end{minipage}&\begin{minipage}[c][\blankheight]{0pt}\end{minipage}&$
\begin{array}{c}
\scriptstyle{5.38829} \\[-6pt]
\scriptscriptstyle{\pm0.223526}
\end{array}
$
\noindent\parbox[p]{4ex}{\renewcommand{\sparklineheight}{2.75}
\begin{sparkline}{4}
\sparkspike 0.10 0.30
\sparkspike 0.20 0.10
\sparkspike 0.30 0.00
\sparkspike 0.40 0.00
\sparkspike 0.50 0.00
\definecolor{sparkspikecolor}{named}{red}
\sparkspike 0.60 0.10
\definecolor{sparkspikecolor}{named}{black}
\sparkspike 0.70 0.10
\sparkspike 0.80 0.10
\sparkspike 0.90 0.20
\sparkspike 1.00 0.10
\sparkbottomline
\end{sparkline}
\renewcommand{\sparklineheight}{1.75}}
\\ 
dec-tree&\begin{minipage}[c][\blankheight]{0pt}\end{minipage}&&\multicolumn{1}{l}{\warmup}&$
\begin{array}{c}
\scriptstyle{1230.5} \\[-6pt]
\scriptscriptstyle{(1188.9, 1290.2)}
\end{array}
$
\noindent\parbox[p]{4ex}{\renewcommand{\sparklineheight}{2.75}
\begin{sparkline}{4}
\sparkspike 0.10 0.20
\sparkspike 0.20 0.20
\sparkspike 0.30 0.00
\definecolor{sparkspikecolor}{named}{red}
\sparkspike 0.40 0.10
\definecolor{sparkspikecolor}{named}{black}
\sparkspike 0.50 0.10
\sparkspike 0.60 0.00
\sparkspike 0.70 0.10
\sparkspike 0.80 0.10
\sparkspike 0.90 0.00
\sparkspike 1.00 0.20
\sparkbottomline
\end{sparkline}
\renewcommand{\sparklineheight}{1.75}}
&$
\begin{array}{c}
\scriptstyle{1968.69} \\[-6pt]
\scriptscriptstyle{(1922.469, 2021.127)}
\end{array}
$
\noindent\parbox[p]{4ex}{\renewcommand{\sparklineheight}{2.75}
\begin{sparkline}{4}
\sparkspike 0.10 0.30
\sparkspike 0.20 0.00
\sparkspike 0.30 0.10
\sparkspike 0.40 0.00
\definecolor{sparkspikecolor}{named}{red}
\sparkspike 0.50 0.30
\definecolor{sparkspikecolor}{named}{black}
\sparkspike 0.60 0.10
\sparkspike 0.70 0.00
\sparkspike 0.80 0.00
\sparkspike 0.90 0.10
\sparkspike 1.00 0.10
\sparkbottomline
\end{sparkline}
\renewcommand{\sparklineheight}{1.75}}
&$
\begin{array}{c}
\scriptstyle{1.55375} \\[-6pt]
\scriptscriptstyle{\pm0.054750}
\end{array}
$
\noindent\parbox[p]{4ex}{\renewcommand{\sparklineheight}{2.75}
\begin{sparkline}{4}
\sparkspike 0.10 0.40
\definecolor{sparkspikecolor}{named}{red}
\sparkspike 0.20 0.20
\definecolor{sparkspikecolor}{named}{black}
\sparkspike 0.30 0.00
\sparkspike 0.40 0.00
\sparkspike 0.50 0.00
\sparkspike 0.60 0.00
\sparkspike 0.70 0.10
\sparkspike 0.80 0.10
\sparkspike 0.90 0.10
\sparkspike 1.00 0.10
\sparkbottomline
\end{sparkline}
\renewcommand{\sparklineheight}{1.75}}
\\ 
dotty&\begin{minipage}[c][\blankheight]{0pt}\end{minipage}&&\multicolumn{1}{l}{\warmup}&$
\begin{array}{c}
\scriptstyle{279.0} \\[-6pt]
\scriptscriptstyle{(250.8, 322.6)}
\end{array}
$
\noindent\parbox[p]{4ex}{\renewcommand{\sparklineheight}{2.75}
\begin{sparkline}{4}
\sparkspike 0.10 0.10
\sparkspike 0.20 0.10
\sparkspike 0.30 0.00
\sparkspike 0.40 0.10
\definecolor{sparkspikecolor}{named}{red}
\sparkspike 0.50 0.30
\definecolor{sparkspikecolor}{named}{black}
\sparkspike 0.60 0.00
\sparkspike 0.70 0.00
\sparkspike 0.80 0.20
\sparkspike 0.90 0.00
\sparkspike 1.00 0.20
\sparkbottomline
\end{sparkline}
\renewcommand{\sparklineheight}{1.75}}
&$
\begin{array}{c}
\scriptstyle{316.56} \\[-6pt]
\scriptscriptstyle{(286.880, 364.796)}
\end{array}
$
\noindent\parbox[p]{4ex}{\renewcommand{\sparklineheight}{2.75}
\begin{sparkline}{4}
\sparkspike 0.10 0.10
\sparkspike 0.20 0.00
\sparkspike 0.30 0.10
\sparkspike 0.40 0.10
\definecolor{sparkspikecolor}{named}{red}
\sparkspike 0.50 0.30
\definecolor{sparkspikecolor}{named}{black}
\sparkspike 0.60 0.00
\sparkspike 0.70 0.00
\sparkspike 0.80 0.20
\sparkspike 0.90 0.00
\sparkspike 1.00 0.20
\sparkbottomline
\end{sparkline}
\renewcommand{\sparklineheight}{1.75}}
&$
\begin{array}{c}
\scriptstyle{1.05536} \\[-6pt]
\scriptscriptstyle{\pm0.008675}
\end{array}
$
\noindent\parbox[p]{4ex}{\renewcommand{\sparklineheight}{2.75}
\begin{sparkline}{4}
\sparkspike 0.10 0.10
\sparkspike 0.20 0.00
\sparkspike 0.30 0.30
\definecolor{sparkspikecolor}{named}{red}
\sparkspike 0.40 0.20
\definecolor{sparkspikecolor}{named}{black}
\sparkspike 0.50 0.00
\sparkspike 0.60 0.10
\sparkspike 0.70 0.10
\sparkspike 0.80 0.10
\sparkspike 0.90 0.00
\sparkspike 1.00 0.10
\sparkbottomline
\end{sparkline}
\renewcommand{\sparklineheight}{1.75}}
\\ 
fj-kmeans&\begin{minipage}[c][\blankheight]{0pt}\end{minipage}&&\multicolumn{1}{l}{\goodinconsistent \scriptsize($8$\warmup, $2$\flatc)}&$
\begin{array}{c}
\scriptstyle{2.0} \\[-6pt]
\scriptscriptstyle{(1.0, 2.5)}
\end{array}
$
\noindent\parbox[p]{4ex}{\renewcommand{\sparklineheight}{2.75}
\begin{sparkline}{4}
\sparkspike 0.10 0.20
\sparkspike 0.20 0.00
\sparkspike 0.30 0.00
\sparkspike 0.40 0.00
\sparkspike 0.50 0.00
\definecolor{sparkspikecolor}{named}{red}
\sparkspike 0.60 0.70
\definecolor{sparkspikecolor}{named}{black}
\sparkspike 0.70 0.00
\sparkspike 0.80 0.00
\sparkspike 0.90 0.00
\sparkspike 1.00 0.10
\sparkbottomline
\end{sparkline}
\renewcommand{\sparklineheight}{1.75}}
&$
\begin{array}{c}
\scriptstyle{4.16} \\[-6pt]
\scriptscriptstyle{(0.000, 6.472)}
\end{array}
$
\noindent\parbox[p]{4ex}{\renewcommand{\sparklineheight}{2.75}
\begin{sparkline}{4}
\sparkspike 0.10 0.20
\sparkspike 0.20 0.00
\sparkspike 0.30 0.00
\sparkspike 0.40 0.00
\sparkspike 0.50 0.00
\definecolor{sparkspikecolor}{named}{red}
\sparkspike 0.60 0.70
\definecolor{sparkspikecolor}{named}{black}
\sparkspike 0.70 0.00
\sparkspike 0.80 0.00
\sparkspike 0.90 0.00
\sparkspike 1.00 0.10
\sparkbottomline
\end{sparkline}
\renewcommand{\sparklineheight}{1.75}}
&$
\begin{array}{c}
\scriptstyle{3.43671} \\[-6pt]
\scriptscriptstyle{\pm0.372966}
\end{array}
$
\noindent\parbox[p]{4ex}{\renewcommand{\sparklineheight}{2.75}
\begin{sparkline}{4}
\sparkspike 0.10 0.20
\sparkspike 0.20 0.20
\definecolor{sparkspikecolor}{named}{red}
\sparkspike 0.30 0.10
\definecolor{sparkspikecolor}{named}{black}
\sparkspike 0.40 0.00
\sparkspike 0.50 0.00
\sparkspike 0.60 0.00
\sparkspike 0.70 0.10
\sparkspike 0.80 0.20
\sparkspike 0.90 0.00
\sparkspike 1.00 0.20
\sparkbottomline
\end{sparkline}
\renewcommand{\sparklineheight}{1.75}}
\\ 
future-genetic&\begin{minipage}[c][\blankheight]{0pt}\end{minipage}&&\multicolumn{1}{l}{\badinconsistent \scriptsize($5$\warmup, $4$\flatc, $1$\slowdown)}&$
\begin{array}{c}
\scriptstyle{14.0} \\[-6pt]
\scriptscriptstyle{(1.0, 987.9)}
\end{array}
$
\noindent\parbox[p]{4ex}{\renewcommand{\sparklineheight}{2.75}
\begin{sparkline}{4}
\definecolor{sparkspikecolor}{named}{red}
\sparkspike 0.10 0.70
\definecolor{sparkspikecolor}{named}{black}
\sparkspike 0.20 0.00
\sparkspike 0.30 0.00
\sparkspike 0.40 0.10
\sparkspike 0.50 0.00
\sparkspike 0.60 0.10
\sparkspike 0.70 0.00
\sparkspike 0.80 0.00
\sparkspike 0.90 0.00
\sparkspike 1.00 0.10
\sparkbottomline
\end{sparkline}
\renewcommand{\sparklineheight}{1.75}}
&$
\begin{array}{c}
\scriptstyle{20.01} \\[-6pt]
\scriptscriptstyle{(0.000, 1503.241)}
\end{array}
$
\noindent\parbox[p]{4ex}{\renewcommand{\sparklineheight}{2.75}
\begin{sparkline}{4}
\definecolor{sparkspikecolor}{named}{red}
\sparkspike 0.10 0.70
\definecolor{sparkspikecolor}{named}{black}
\sparkspike 0.20 0.00
\sparkspike 0.30 0.00
\sparkspike 0.40 0.10
\sparkspike 0.50 0.00
\sparkspike 0.60 0.10
\sparkspike 0.70 0.00
\sparkspike 0.80 0.00
\sparkspike 0.90 0.00
\sparkspike 1.00 0.10
\sparkbottomline
\end{sparkline}
\renewcommand{\sparklineheight}{1.75}}
&$
\begin{array}{c}
\scriptstyle{1.51295} \\[-6pt]
\scriptscriptstyle{\pm0.058047}
\end{array}
$
\noindent\parbox[p]{4ex}{\renewcommand{\sparklineheight}{2.75}
\begin{sparkline}{4}
\sparkspike 0.10 0.20
\sparkspike 0.20 0.00
\sparkspike 0.30 0.10
\sparkspike 0.40 0.10
\sparkspike 0.50 0.00
\definecolor{sparkspikecolor}{named}{red}
\sparkspike 0.60 0.40
\definecolor{sparkspikecolor}{named}{black}
\sparkspike 0.70 0.10
\sparkspike 0.80 0.00
\sparkspike 0.90 0.00
\sparkspike 1.00 0.10
\sparkbottomline
\end{sparkline}
\renewcommand{\sparklineheight}{1.75}}
\\ 
gauss-mix&\begin{minipage}[c][\blankheight]{0pt}\end{minipage}&&\multicolumn{1}{l}{\warmup}&$
\begin{array}{c}
\scriptstyle{161.0} \\[-6pt]
\scriptscriptstyle{(156.5, 237.6)}
\end{array}
$
\noindent\parbox[p]{4ex}{\renewcommand{\sparklineheight}{2.75}
\begin{sparkline}{4}
\definecolor{sparkspikecolor}{named}{red}
\sparkspike 0.10 0.80
\definecolor{sparkspikecolor}{named}{black}
\sparkspike 0.20 0.00
\sparkspike 0.30 0.10
\sparkspike 0.40 0.00
\sparkspike 0.50 0.00
\sparkspike 0.60 0.00
\sparkspike 0.70 0.00
\sparkspike 0.80 0.00
\sparkspike 0.90 0.00
\sparkspike 1.00 0.10
\sparkbottomline
\end{sparkline}
\renewcommand{\sparklineheight}{1.75}}
&$
\begin{array}{c}
\scriptstyle{140.39} \\[-6pt]
\scriptscriptstyle{(130.610, 204.081)}
\end{array}
$
\noindent\parbox[p]{4ex}{\renewcommand{\sparklineheight}{2.75}
\begin{sparkline}{4}
\sparkspike 0.10 0.20
\definecolor{sparkspikecolor}{named}{red}
\sparkspike 0.20 0.50
\definecolor{sparkspikecolor}{named}{black}
\sparkspike 0.30 0.20
\sparkspike 0.40 0.00
\sparkspike 0.50 0.00
\sparkspike 0.60 0.00
\sparkspike 0.70 0.00
\sparkspike 0.80 0.00
\sparkspike 0.90 0.00
\sparkspike 1.00 0.10
\sparkbottomline
\end{sparkline}
\renewcommand{\sparklineheight}{1.75}}
&$
\begin{array}{c}
\scriptstyle{0.83647} \\[-6pt]
\scriptscriptstyle{\pm0.037775}
\end{array}
$
\noindent\parbox[p]{4ex}{\renewcommand{\sparklineheight}{2.75}
\begin{sparkline}{4}
\sparkspike 0.10 0.10
\sparkspike 0.20 0.00
\sparkspike 0.30 0.00
\sparkspike 0.40 0.00
\sparkspike 0.50 0.00
\sparkspike 0.60 0.00
\sparkspike 0.70 0.30
\sparkspike 0.80 0.00
\definecolor{sparkspikecolor}{named}{red}
\sparkspike 0.90 0.30
\definecolor{sparkspikecolor}{named}{black}
\sparkspike 1.00 0.30
\sparkbottomline
\end{sparkline}
\renewcommand{\sparklineheight}{1.75}}
\\ 
log-regression&\begin{minipage}[c][\blankheight]{0pt}\end{minipage}&&\multicolumn{1}{l}{\badinconsistent \scriptsize($9$\nosteadystate, $1$\warmup)}&\begin{minipage}[c][\blankheight]{0pt}\end{minipage}&\begin{minipage}[c][\blankheight]{0pt}\end{minipage}&\begin{minipage}[c][\blankheight]{0pt}\end{minipage}\\ 
mnemonics&\begin{minipage}[c][\blankheight]{0pt}\end{minipage}&&\multicolumn{1}{l}{\badinconsistent \scriptsize($8$\warmup, $2$\slowdown)}&$
\begin{array}{c}
\scriptstyle{3.0} \\[-6pt]
\scriptscriptstyle{(3.0, 398.1)}
\end{array}
$
\noindent\parbox[p]{4ex}{\renewcommand{\sparklineheight}{2.75}
\begin{sparkline}{4}
\definecolor{sparkspikecolor}{named}{red}
\sparkspike 0.10 0.70
\definecolor{sparkspikecolor}{named}{black}
\sparkspike 0.20 0.00
\sparkspike 0.30 0.00
\sparkspike 0.40 0.00
\sparkspike 0.50 0.00
\sparkspike 0.60 0.00
\sparkspike 0.70 0.00
\sparkspike 0.80 0.00
\sparkspike 0.90 0.10
\sparkspike 1.00 0.20
\sparkbottomline
\end{sparkline}
\renewcommand{\sparklineheight}{1.75}}
&$
\begin{array}{c}
\scriptstyle{7.21} \\[-6pt]
\scriptscriptstyle{(7.089, 1215.030)}
\end{array}
$
\noindent\parbox[p]{4ex}{\renewcommand{\sparklineheight}{2.75}
\begin{sparkline}{4}
\definecolor{sparkspikecolor}{named}{red}
\sparkspike 0.10 0.70
\definecolor{sparkspikecolor}{named}{black}
\sparkspike 0.20 0.00
\sparkspike 0.30 0.00
\sparkspike 0.40 0.00
\sparkspike 0.50 0.00
\sparkspike 0.60 0.00
\sparkspike 0.70 0.00
\sparkspike 0.80 0.00
\sparkspike 0.90 0.00
\sparkspike 1.00 0.30
\sparkbottomline
\end{sparkline}
\renewcommand{\sparklineheight}{1.75}}
&$
\begin{array}{c}
\scriptstyle{3.02920} \\[-6pt]
\scriptscriptstyle{\pm0.079644}
\end{array}
$
\noindent\parbox[p]{4ex}{\renewcommand{\sparklineheight}{2.75}
\begin{sparkline}{4}
\sparkspike 0.10 0.20
\sparkspike 0.20 0.20
\definecolor{sparkspikecolor}{named}{red}
\sparkspike 0.30 0.20
\definecolor{sparkspikecolor}{named}{black}
\sparkspike 0.40 0.00
\sparkspike 0.50 0.00
\sparkspike 0.60 0.20
\sparkspike 0.70 0.10
\sparkspike 0.80 0.00
\sparkspike 0.90 0.00
\sparkspike 1.00 0.10
\sparkbottomline
\end{sparkline}
\renewcommand{\sparklineheight}{1.75}}
\\ 
naive-bayes&\begin{minipage}[c][\blankheight]{0pt}\end{minipage}&&\multicolumn{1}{l}{\badinconsistent \scriptsize($7$\warmup, $2$\nosteadystate, $1$\slowdown)}&\begin{minipage}[c][\blankheight]{0pt}\end{minipage}&\begin{minipage}[c][\blankheight]{0pt}\end{minipage}&\begin{minipage}[c][\blankheight]{0pt}\end{minipage}\\ 
neo4j-analytics&\begin{minipage}[c][\blankheight]{0pt}\end{minipage}&&\multicolumn{1}{l}{\badinconsistent \scriptsize($5$\flatc, $3$\warmup, $1$\slowdown, $1$\nosteadystate)}&\begin{minipage}[c][\blankheight]{0pt}\end{minipage}&\begin{minipage}[c][\blankheight]{0pt}\end{minipage}&\begin{minipage}[c][\blankheight]{0pt}\end{minipage}\\ 
par-mnemonics&\begin{minipage}[c][\blankheight]{0pt}\end{minipage}&&\multicolumn{1}{l}{\badinconsistent \scriptsize($7$\warmup, $3$\slowdown)}&$
\begin{array}{c}
\scriptstyle{221.0} \\[-6pt]
\scriptscriptstyle{(4.5, 607.5)}
\end{array}
$
\noindent\parbox[p]{4ex}{\renewcommand{\sparklineheight}{2.75}
\begin{sparkline}{4}
\sparkspike 0.10 0.20
\sparkspike 0.20 0.20
\definecolor{sparkspikecolor}{named}{red}
\sparkspike 0.30 0.10
\definecolor{sparkspikecolor}{named}{black}
\sparkspike 0.40 0.20
\sparkspike 0.50 0.10
\sparkspike 0.60 0.00
\sparkspike 0.70 0.00
\sparkspike 0.80 0.00
\sparkspike 0.90 0.10
\sparkspike 1.00 0.10
\sparkbottomline
\end{sparkline}
\renewcommand{\sparklineheight}{1.75}}
&$
\begin{array}{c}
\scriptstyle{573.17} \\[-6pt]
\scriptscriptstyle{(10.387, 1611.834)}
\end{array}
$
\noindent\parbox[p]{4ex}{\renewcommand{\sparklineheight}{2.75}
\begin{sparkline}{4}
\sparkspike 0.10 0.20
\sparkspike 0.20 0.20
\definecolor{sparkspikecolor}{named}{red}
\sparkspike 0.30 0.10
\definecolor{sparkspikecolor}{named}{black}
\sparkspike 0.40 0.20
\sparkspike 0.50 0.10
\sparkspike 0.60 0.00
\sparkspike 0.70 0.00
\sparkspike 0.80 0.00
\sparkspike 0.90 0.10
\sparkspike 1.00 0.10
\sparkbottomline
\end{sparkline}
\renewcommand{\sparklineheight}{1.75}}
&$
\begin{array}{c}
\scriptstyle{2.64939} \\[-6pt]
\scriptscriptstyle{\pm0.120870}
\end{array}
$
\noindent\parbox[p]{4ex}{\renewcommand{\sparklineheight}{2.75}
\begin{sparkline}{4}
\sparkspike 0.10 0.20
\sparkspike 0.20 0.00
\sparkspike 0.30 0.00
\sparkspike 0.40 0.10
\definecolor{sparkspikecolor}{named}{red}
\sparkspike 0.50 0.20
\definecolor{sparkspikecolor}{named}{black}
\sparkspike 0.60 0.20
\sparkspike 0.70 0.10
\sparkspike 0.80 0.00
\sparkspike 0.90 0.10
\sparkspike 1.00 0.10
\sparkbottomline
\end{sparkline}
\renewcommand{\sparklineheight}{1.75}}
\\ 
philosophers&\begin{minipage}[c][\blankheight]{0pt}\end{minipage}&&\multicolumn{1}{l}{\badinconsistent \scriptsize($9$\slowdown, $1$\nosteadystate)}&\begin{minipage}[c][\blankheight]{0pt}\end{minipage}&\begin{minipage}[c][\blankheight]{0pt}\end{minipage}&\begin{minipage}[c][\blankheight]{0pt}\end{minipage}\\ 
reactors&\begin{minipage}[c][\blankheight]{0pt}\end{minipage}&&\multicolumn{1}{l}{\badinconsistent \scriptsize($4$\flatc, $3$\warmup, $3$\slowdown)}&$
\begin{array}{c}
\scriptstyle{2.0} \\[-6pt]
\scriptscriptstyle{(1.0, 931.7)}
\end{array}
$
\noindent\parbox[p]{4ex}{\renewcommand{\sparklineheight}{2.75}
\begin{sparkline}{4}
\definecolor{sparkspikecolor}{named}{red}
\sparkspike 0.10 0.70
\definecolor{sparkspikecolor}{named}{black}
\sparkspike 0.20 0.00
\sparkspike 0.30 0.00
\sparkspike 0.40 0.00
\sparkspike 0.50 0.00
\sparkspike 0.60 0.00
\sparkspike 0.70 0.00
\sparkspike 0.80 0.20
\sparkspike 0.90 0.00
\sparkspike 1.00 0.10
\sparkbottomline
\end{sparkline}
\renewcommand{\sparklineheight}{1.75}}
&$
\begin{array}{c}
\scriptstyle{11.09} \\[-6pt]
\scriptscriptstyle{(0.000, 8442.628)}
\end{array}
$
\noindent\parbox[p]{4ex}{\renewcommand{\sparklineheight}{2.75}
\begin{sparkline}{4}
\definecolor{sparkspikecolor}{named}{red}
\sparkspike 0.10 0.70
\definecolor{sparkspikecolor}{named}{black}
\sparkspike 0.20 0.00
\sparkspike 0.30 0.00
\sparkspike 0.40 0.00
\sparkspike 0.50 0.00
\sparkspike 0.60 0.00
\sparkspike 0.70 0.00
\sparkspike 0.80 0.20
\sparkspike 0.90 0.00
\sparkspike 1.00 0.10
\sparkbottomline
\end{sparkline}
\renewcommand{\sparklineheight}{1.75}}
&$
\begin{array}{c}
\scriptstyle{9.17997} \\[-6pt]
\scriptscriptstyle{\pm0.107281}
\end{array}
$
\noindent\parbox[p]{4ex}{\renewcommand{\sparklineheight}{2.75}
\begin{sparkline}{4}
\sparkspike 0.10 0.10
\sparkspike 0.20 0.10
\sparkspike 0.30 0.00
\sparkspike 0.40 0.00
\sparkspike 0.50 0.00
\sparkspike 0.60 0.20
\definecolor{sparkspikecolor}{named}{red}
\sparkspike 0.70 0.10
\definecolor{sparkspikecolor}{named}{black}
\sparkspike 0.80 0.20
\sparkspike 0.90 0.10
\sparkspike 1.00 0.20
\sparkbottomline
\end{sparkline}
\renewcommand{\sparklineheight}{1.75}}
\\ 
rx-scrabble&\begin{minipage}[c][\blankheight]{0pt}\end{minipage}&&\multicolumn{1}{l}{\badinconsistent \scriptsize($6$\slowdown, $3$\warmup, $1$\nosteadystate)}&\begin{minipage}[c][\blankheight]{0pt}\end{minipage}&\begin{minipage}[c][\blankheight]{0pt}\end{minipage}&\begin{minipage}[c][\blankheight]{0pt}\end{minipage}\\ 
scala-kmeans&\begin{minipage}[c][\blankheight]{0pt}\end{minipage}&&\multicolumn{1}{l}{\badinconsistent \scriptsize($4$\nosteadystate, $4$\warmup, $2$\slowdown)}&\begin{minipage}[c][\blankheight]{0pt}\end{minipage}&\begin{minipage}[c][\blankheight]{0pt}\end{minipage}&\begin{minipage}[c][\blankheight]{0pt}\end{minipage}\\ 
scala-stm-bench7&\begin{minipage}[c][\blankheight]{0pt}\end{minipage}&&\multicolumn{1}{l}{\warmup}&$
\begin{array}{c}
\scriptstyle{264.0} \\[-6pt]
\scriptscriptstyle{(117.8, 910.0)}
\end{array}
$
\noindent\parbox[p]{4ex}{\renewcommand{\sparklineheight}{2.75}
\begin{sparkline}{4}
\sparkspike 0.10 0.30
\definecolor{sparkspikecolor}{named}{red}
\sparkspike 0.20 0.60
\definecolor{sparkspikecolor}{named}{black}
\sparkspike 0.30 0.00
\sparkspike 0.40 0.00
\sparkspike 0.50 0.00
\sparkspike 0.60 0.00
\sparkspike 0.70 0.00
\sparkspike 0.80 0.00
\sparkspike 0.90 0.00
\sparkspike 1.00 0.10
\sparkbottomline
\end{sparkline}
\renewcommand{\sparklineheight}{1.75}}
&$
\begin{array}{c}
\scriptstyle{206.91} \\[-6pt]
\scriptscriptstyle{(95.792, 710.094)}
\end{array}
$
\noindent\parbox[p]{4ex}{\renewcommand{\sparklineheight}{2.75}
\begin{sparkline}{4}
\sparkspike 0.10 0.30
\definecolor{sparkspikecolor}{named}{red}
\sparkspike 0.20 0.60
\definecolor{sparkspikecolor}{named}{black}
\sparkspike 0.30 0.00
\sparkspike 0.40 0.00
\sparkspike 0.50 0.00
\sparkspike 0.60 0.00
\sparkspike 0.70 0.00
\sparkspike 0.80 0.00
\sparkspike 0.90 0.00
\sparkspike 1.00 0.10
\sparkbottomline
\end{sparkline}
\renewcommand{\sparklineheight}{1.75}}
&$
\begin{array}{c}
\scriptstyle{0.76991} \\[-6pt]
\scriptscriptstyle{\pm0.010929}
\end{array}
$
\noindent\parbox[p]{4ex}{\renewcommand{\sparklineheight}{2.75}
\begin{sparkline}{4}
\sparkspike 0.10 0.30
\sparkspike 0.20 0.10
\definecolor{sparkspikecolor}{named}{red}
\sparkspike 0.30 0.10
\definecolor{sparkspikecolor}{named}{black}
\sparkspike 0.40 0.00
\sparkspike 0.50 0.20
\sparkspike 0.60 0.20
\sparkspike 0.70 0.00
\sparkspike 0.80 0.00
\sparkspike 0.90 0.00
\sparkspike 1.00 0.10
\sparkbottomline
\end{sparkline}
\renewcommand{\sparklineheight}{1.75}}
\\ 
scrabble&\begin{minipage}[c][\blankheight]{0pt}\end{minipage}&&\multicolumn{1}{l}{\goodinconsistent \scriptsize($9$\flatc, $1$\warmup)}&$
\begin{array}{c}
\scriptstyle{1.0} \\[-6pt]
\scriptscriptstyle{(1.0, 49.9)}
\end{array}
$
\noindent\parbox[p]{4ex}{\renewcommand{\sparklineheight}{2.75}
\begin{sparkline}{4}
\definecolor{sparkspikecolor}{named}{red}
\sparkspike 0.10 0.90
\definecolor{sparkspikecolor}{named}{black}
\sparkspike 0.20 0.00
\sparkspike 0.30 0.00
\sparkspike 0.40 0.00
\sparkspike 0.50 0.00
\sparkspike 0.60 0.00
\sparkspike 0.70 0.00
\sparkspike 0.80 0.00
\sparkspike 0.90 0.00
\sparkspike 1.00 0.10
\sparkbottomline
\end{sparkline}
\renewcommand{\sparklineheight}{1.75}}
&$
\begin{array}{c}
\scriptstyle{0.00} \\[-6pt]
\scriptscriptstyle{(0.000, 15.530)}
\end{array}
$
\noindent\parbox[p]{4ex}{\renewcommand{\sparklineheight}{2.75}
\begin{sparkline}{4}
\definecolor{sparkspikecolor}{named}{red}
\sparkspike 0.10 0.90
\definecolor{sparkspikecolor}{named}{black}
\sparkspike 0.20 0.00
\sparkspike 0.30 0.00
\sparkspike 0.40 0.00
\sparkspike 0.50 0.00
\sparkspike 0.60 0.00
\sparkspike 0.70 0.00
\sparkspike 0.80 0.00
\sparkspike 0.90 0.00
\sparkspike 1.00 0.10
\sparkbottomline
\end{sparkline}
\renewcommand{\sparklineheight}{1.75}}
&$
\begin{array}{c}
\scriptstyle{0.29152} \\[-6pt]
\scriptscriptstyle{\pm0.038712}
\end{array}
$
\noindent\parbox[p]{4ex}{\renewcommand{\sparklineheight}{2.75}
\begin{sparkline}{4}
\sparkspike 0.10 0.10
\sparkspike 0.20 0.00
\sparkspike 0.30 0.00
\sparkspike 0.40 0.00
\sparkspike 0.50 0.10
\sparkspike 0.60 0.10
\sparkspike 0.70 0.00
\definecolor{sparkspikecolor}{named}{red}
\sparkspike 0.80 0.20
\definecolor{sparkspikecolor}{named}{black}
\sparkspike 0.90 0.40
\sparkspike 1.00 0.10
\sparkbottomline
\end{sparkline}
\renewcommand{\sparklineheight}{1.75}}
\\ 
\hline
akka-uct&\begin{minipage}[c][\blankheight]{0pt}\end{minipage}&\multirow{20}{*}{\rotatebox[origin=c]{90}{graal-ce-hotspot}}&\multicolumn{1}{l}{\goodinconsistent \scriptsize($7$\flatc, $3$\warmup)}&$
\begin{array}{c}
\scriptstyle{1.0} \\[-6pt]
\scriptscriptstyle{(1.0, 572.6)}
\end{array}
$
\noindent\parbox[p]{4ex}{\renewcommand{\sparklineheight}{2.75}
\begin{sparkline}{4}
\definecolor{sparkspikecolor}{named}{red}
\sparkspike 0.10 0.80
\definecolor{sparkspikecolor}{named}{black}
\sparkspike 0.20 0.00
\sparkspike 0.30 0.00
\sparkspike 0.40 0.00
\sparkspike 0.50 0.00
\sparkspike 0.60 0.00
\sparkspike 0.70 0.00
\sparkspike 0.80 0.00
\sparkspike 0.90 0.10
\sparkspike 1.00 0.10
\sparkbottomline
\end{sparkline}
\renewcommand{\sparklineheight}{1.75}}
&$
\begin{array}{c}
\scriptstyle{0.00} \\[-6pt]
\scriptscriptstyle{(0.000, 4197.425)}
\end{array}
$
\noindent\parbox[p]{4ex}{\renewcommand{\sparklineheight}{2.75}
\begin{sparkline}{4}
\definecolor{sparkspikecolor}{named}{red}
\sparkspike 0.10 0.80
\definecolor{sparkspikecolor}{named}{black}
\sparkspike 0.20 0.00
\sparkspike 0.30 0.00
\sparkspike 0.40 0.00
\sparkspike 0.50 0.00
\sparkspike 0.60 0.00
\sparkspike 0.70 0.00
\sparkspike 0.80 0.00
\sparkspike 0.90 0.10
\sparkspike 1.00 0.10
\sparkbottomline
\end{sparkline}
\renewcommand{\sparklineheight}{1.75}}
&$
\begin{array}{c}
\scriptstyle{7.24920} \\[-6pt]
\scriptscriptstyle{\pm0.083997}
\end{array}
$
\noindent\parbox[p]{4ex}{\renewcommand{\sparklineheight}{2.75}
\begin{sparkline}{4}
\definecolor{sparkspikecolor}{named}{red}
\sparkspike 0.10 0.60
\definecolor{sparkspikecolor}{named}{black}
\sparkspike 0.20 0.20
\sparkspike 0.30 0.00
\sparkspike 0.40 0.00
\sparkspike 0.50 0.00
\sparkspike 0.60 0.00
\sparkspike 0.70 0.00
\sparkspike 0.80 0.10
\sparkspike 0.90 0.00
\sparkspike 1.00 0.10
\sparkbottomline
\end{sparkline}
\renewcommand{\sparklineheight}{1.75}}
\\ 
als&\begin{minipage}[c][\blankheight]{0pt}\end{minipage}&&\multicolumn{1}{l}{\badinconsistent \scriptsize($5$\nosteadystate, $3$\warmup, $1$\slowdown, $1$\flatc)}&\begin{minipage}[c][\blankheight]{0pt}\end{minipage}&\begin{minipage}[c][\blankheight]{0pt}\end{minipage}&\begin{minipage}[c][\blankheight]{0pt}\end{minipage}\\ 
chi-square&\begin{minipage}[c][\blankheight]{0pt}\end{minipage}&&\multicolumn{1}{l}{\goodinconsistent \scriptsize($5$\flatc, $5$\warmup)}&$
\begin{array}{c}
\scriptstyle{72.0} \\[-6pt]
\scriptscriptstyle{(1.0, 332.4)}
\end{array}
$
\noindent\parbox[p]{4ex}{\renewcommand{\sparklineheight}{2.75}
\begin{sparkline}{4}
\definecolor{sparkspikecolor}{named}{red}
\sparkspike 0.10 0.50
\definecolor{sparkspikecolor}{named}{black}
\sparkspike 0.20 0.00
\sparkspike 0.30 0.00
\sparkspike 0.40 0.10
\sparkspike 0.50 0.20
\sparkspike 0.60 0.10
\sparkspike 0.70 0.00
\sparkspike 0.80 0.00
\sparkspike 0.90 0.00
\sparkspike 1.00 0.10
\sparkbottomline
\end{sparkline}
\renewcommand{\sparklineheight}{1.75}}
&$
\begin{array}{c}
\scriptstyle{82.60} \\[-6pt]
\scriptscriptstyle{(0.000, 315.061)}
\end{array}
$
\noindent\parbox[p]{4ex}{\renewcommand{\sparklineheight}{2.75}
\begin{sparkline}{4}
\definecolor{sparkspikecolor}{named}{red}
\sparkspike 0.10 0.50
\definecolor{sparkspikecolor}{named}{black}
\sparkspike 0.20 0.00
\sparkspike 0.30 0.00
\sparkspike 0.40 0.00
\sparkspike 0.50 0.10
\sparkspike 0.60 0.20
\sparkspike 0.70 0.00
\sparkspike 0.80 0.10
\sparkspike 0.90 0.00
\sparkspike 1.00 0.10
\sparkbottomline
\end{sparkline}
\renewcommand{\sparklineheight}{1.75}}
&$
\begin{array}{c}
\scriptstyle{0.92334} \\[-6pt]
\scriptscriptstyle{\pm0.208808}
\end{array}
$
\noindent\parbox[p]{4ex}{\renewcommand{\sparklineheight}{2.75}
\begin{sparkline}{4}
\sparkspike 0.10 0.20
\sparkspike 0.20 0.10
\sparkspike 0.30 0.10
\definecolor{sparkspikecolor}{named}{red}
\sparkspike 0.40 0.20
\definecolor{sparkspikecolor}{named}{black}
\sparkspike 0.50 0.00
\sparkspike 0.60 0.00
\sparkspike 0.70 0.10
\sparkspike 0.80 0.10
\sparkspike 0.90 0.10
\sparkspike 1.00 0.10
\sparkbottomline
\end{sparkline}
\renewcommand{\sparklineheight}{1.75}}
\\ 
db-shootout&\begin{minipage}[c][\blankheight]{0pt}\end{minipage}&&\multicolumn{1}{l}{\flatc}&\begin{minipage}[c][\blankheight]{0pt}\end{minipage}&\begin{minipage}[c][\blankheight]{0pt}\end{minipage}&$
\begin{array}{c}
\scriptstyle{6.05240} \\[-6pt]
\scriptscriptstyle{\pm0.265592}
\end{array}
$
\noindent\parbox[p]{4ex}{\renewcommand{\sparklineheight}{2.75}
\begin{sparkline}{4}
\sparkspike 0.10 0.10
\sparkspike 0.20 0.10
\sparkspike 0.30 0.10
\sparkspike 0.40 0.10
\sparkspike 0.50 0.00
\definecolor{sparkspikecolor}{named}{red}
\sparkspike 0.60 0.20
\definecolor{sparkspikecolor}{named}{black}
\sparkspike 0.70 0.00
\sparkspike 0.80 0.20
\sparkspike 0.90 0.00
\sparkspike 1.00 0.20
\sparkbottomline
\end{sparkline}
\renewcommand{\sparklineheight}{1.75}}
\\ 
dec-tree&\begin{minipage}[c][\blankheight]{0pt}\end{minipage}&&\multicolumn{1}{l}{\warmup}&$
\begin{array}{c}
\scriptstyle{1167.0} \\[-6pt]
\scriptscriptstyle{(1084.6, 1228.2)}
\end{array}
$
\noindent\parbox[p]{4ex}{\renewcommand{\sparklineheight}{2.75}
\begin{sparkline}{4}
\sparkspike 0.10 0.10
\sparkspike 0.20 0.10
\sparkspike 0.30 0.00
\sparkspike 0.40 0.00
\sparkspike 0.50 0.20
\definecolor{sparkspikecolor}{named}{red}
\sparkspike 0.60 0.10
\definecolor{sparkspikecolor}{named}{black}
\sparkspike 0.70 0.20
\sparkspike 0.80 0.00
\sparkspike 0.90 0.10
\sparkspike 1.00 0.20
\sparkbottomline
\end{sparkline}
\renewcommand{\sparklineheight}{1.75}}
&$
\begin{array}{c}
\scriptstyle{1950.65} \\[-6pt]
\scriptscriptstyle{(1884.886, 2007.078)}
\end{array}
$
\noindent\parbox[p]{4ex}{\renewcommand{\sparklineheight}{2.75}
\begin{sparkline}{4}
\sparkspike 0.10 0.10
\sparkspike 0.20 0.00
\sparkspike 0.30 0.10
\sparkspike 0.40 0.20
\sparkspike 0.50 0.00
\definecolor{sparkspikecolor}{named}{red}
\sparkspike 0.60 0.30
\definecolor{sparkspikecolor}{named}{black}
\sparkspike 0.70 0.00
\sparkspike 0.80 0.10
\sparkspike 0.90 0.10
\sparkspike 1.00 0.10
\sparkbottomline
\end{sparkline}
\renewcommand{\sparklineheight}{1.75}}
&$
\begin{array}{c}
\scriptstyle{1.66113} \\[-6pt]
\scriptscriptstyle{\pm0.070871}
\end{array}
$
\noindent\parbox[p]{4ex}{\renewcommand{\sparklineheight}{2.75}
\begin{sparkline}{4}
\sparkspike 0.10 0.30
\sparkspike 0.20 0.00
\sparkspike 0.30 0.10
\definecolor{sparkspikecolor}{named}{red}
\sparkspike 0.40 0.10
\definecolor{sparkspikecolor}{named}{black}
\sparkspike 0.50 0.10
\sparkspike 0.60 0.10
\sparkspike 0.70 0.00
\sparkspike 0.80 0.10
\sparkspike 0.90 0.10
\sparkspike 1.00 0.10
\sparkbottomline
\end{sparkline}
\renewcommand{\sparklineheight}{1.75}}
\\ 
dotty&\begin{minipage}[c][\blankheight]{0pt}\end{minipage}&&\multicolumn{1}{l}{\warmup}&$
\begin{array}{c}
\scriptstyle{137.0} \\[-6pt]
\scriptscriptstyle{(104.0, 246.8)}
\end{array}
$
\noindent\parbox[p]{4ex}{\renewcommand{\sparklineheight}{2.75}
\begin{sparkline}{4}
\sparkspike 0.10 0.30
\definecolor{sparkspikecolor}{named}{red}
\sparkspike 0.20 0.20
\definecolor{sparkspikecolor}{named}{black}
\sparkspike 0.30 0.00
\sparkspike 0.40 0.10
\sparkspike 0.50 0.00
\sparkspike 0.60 0.10
\sparkspike 0.70 0.00
\sparkspike 0.80 0.10
\sparkspike 0.90 0.10
\sparkspike 1.00 0.10
\sparkbottomline
\end{sparkline}
\renewcommand{\sparklineheight}{1.75}}
&$
\begin{array}{c}
\scriptstyle{197.58} \\[-6pt]
\scriptscriptstyle{(152.841, 342.581)}
\end{array}
$
\noindent\parbox[p]{4ex}{\renewcommand{\sparklineheight}{2.75}
\begin{sparkline}{4}
\sparkspike 0.10 0.30
\definecolor{sparkspikecolor}{named}{red}
\sparkspike 0.20 0.20
\definecolor{sparkspikecolor}{named}{black}
\sparkspike 0.30 0.00
\sparkspike 0.40 0.10
\sparkspike 0.50 0.00
\sparkspike 0.60 0.10
\sparkspike 0.70 0.00
\sparkspike 0.80 0.10
\sparkspike 0.90 0.10
\sparkspike 1.00 0.10
\sparkbottomline
\end{sparkline}
\renewcommand{\sparklineheight}{1.75}}
&$
\begin{array}{c}
\scriptstyle{1.30432} \\[-6pt]
\scriptscriptstyle{\pm0.037703}
\end{array}
$
\noindent\parbox[p]{4ex}{\renewcommand{\sparklineheight}{2.75}
\begin{sparkline}{4}
\sparkspike 0.10 0.10
\sparkspike 0.20 0.00
\sparkspike 0.30 0.00
\sparkspike 0.40 0.10
\sparkspike 0.50 0.20
\definecolor{sparkspikecolor}{named}{red}
\sparkspike 0.60 0.20
\definecolor{sparkspikecolor}{named}{black}
\sparkspike 0.70 0.10
\sparkspike 0.80 0.20
\sparkspike 0.90 0.00
\sparkspike 1.00 0.10
\sparkbottomline
\end{sparkline}
\renewcommand{\sparklineheight}{1.75}}
\\ 
fj-kmeans&\begin{minipage}[c][\blankheight]{0pt}\end{minipage}&&\multicolumn{1}{l}{\goodinconsistent \scriptsize($8$\warmup, $2$\flatc)}&$
\begin{array}{c}
\scriptstyle{2.0} \\[-6pt]
\scriptscriptstyle{(1.0, 3.6)}
\end{array}
$
\noindent\parbox[p]{4ex}{\renewcommand{\sparklineheight}{2.75}
\begin{sparkline}{4}
\sparkspike 0.10 0.20
\sparkspike 0.20 0.00
\definecolor{sparkspikecolor}{named}{red}
\sparkspike 0.30 0.70
\definecolor{sparkspikecolor}{named}{black}
\sparkspike 0.40 0.00
\sparkspike 0.50 0.00
\sparkspike 0.60 0.00
\sparkspike 0.70 0.00
\sparkspike 0.80 0.00
\sparkspike 0.90 0.00
\sparkspike 1.00 0.10
\sparkbottomline
\end{sparkline}
\renewcommand{\sparklineheight}{1.75}}
&$
\begin{array}{c}
\scriptstyle{4.11} \\[-6pt]
\scriptscriptstyle{(0.000, 9.880)}
\end{array}
$
\noindent\parbox[p]{4ex}{\renewcommand{\sparklineheight}{2.75}
\begin{sparkline}{4}
\sparkspike 0.10 0.20
\sparkspike 0.20 0.00
\definecolor{sparkspikecolor}{named}{red}
\sparkspike 0.30 0.70
\definecolor{sparkspikecolor}{named}{black}
\sparkspike 0.40 0.00
\sparkspike 0.50 0.00
\sparkspike 0.60 0.00
\sparkspike 0.70 0.00
\sparkspike 0.80 0.00
\sparkspike 0.90 0.00
\sparkspike 1.00 0.10
\sparkbottomline
\end{sparkline}
\renewcommand{\sparklineheight}{1.75}}
&$
\begin{array}{c}
\scriptstyle{3.29264} \\[-6pt]
\scriptscriptstyle{\pm0.360004}
\end{array}
$
\noindent\parbox[p]{4ex}{\renewcommand{\sparklineheight}{2.75}
\begin{sparkline}{4}
\sparkspike 0.10 0.20
\sparkspike 0.20 0.10
\definecolor{sparkspikecolor}{named}{red}
\sparkspike 0.30 0.30
\definecolor{sparkspikecolor}{named}{black}
\sparkspike 0.40 0.10
\sparkspike 0.50 0.10
\sparkspike 0.60 0.10
\sparkspike 0.70 0.00
\sparkspike 0.80 0.00
\sparkspike 0.90 0.00
\sparkspike 1.00 0.10
\sparkbottomline
\end{sparkline}
\renewcommand{\sparklineheight}{1.75}}
\\ 
future-genetic&\begin{minipage}[c][\blankheight]{0pt}\end{minipage}&&\multicolumn{1}{l}{\badinconsistent \scriptsize($9$\warmup, $1$\slowdown)}&$
\begin{array}{c}
\scriptstyle{14.0} \\[-6pt]
\scriptscriptstyle{(2.0, 924.1)}
\end{array}
$
\noindent\parbox[p]{4ex}{\renewcommand{\sparklineheight}{2.75}
\begin{sparkline}{4}
\definecolor{sparkspikecolor}{named}{red}
\sparkspike 0.10 0.70
\definecolor{sparkspikecolor}{named}{black}
\sparkspike 0.20 0.00
\sparkspike 0.30 0.00
\sparkspike 0.40 0.10
\sparkspike 0.50 0.10
\sparkspike 0.60 0.00
\sparkspike 0.70 0.00
\sparkspike 0.80 0.00
\sparkspike 0.90 0.00
\sparkspike 1.00 0.10
\sparkbottomline
\end{sparkline}
\renewcommand{\sparklineheight}{1.75}}
&$
\begin{array}{c}
\scriptstyle{18.88} \\[-6pt]
\scriptscriptstyle{(1.820, 1271.670)}
\end{array}
$
\noindent\parbox[p]{4ex}{\renewcommand{\sparklineheight}{2.75}
\begin{sparkline}{4}
\definecolor{sparkspikecolor}{named}{red}
\sparkspike 0.10 0.70
\definecolor{sparkspikecolor}{named}{black}
\sparkspike 0.20 0.00
\sparkspike 0.30 0.00
\sparkspike 0.40 0.10
\sparkspike 0.50 0.00
\sparkspike 0.60 0.10
\sparkspike 0.70 0.00
\sparkspike 0.80 0.00
\sparkspike 0.90 0.00
\sparkspike 1.00 0.10
\sparkbottomline
\end{sparkline}
\renewcommand{\sparklineheight}{1.75}}
&$
\begin{array}{c}
\scriptstyle{1.41088} \\[-6pt]
\scriptscriptstyle{\pm0.054155}
\end{array}
$
\noindent\parbox[p]{4ex}{\renewcommand{\sparklineheight}{2.75}
\begin{sparkline}{4}
\sparkspike 0.10 0.30
\sparkspike 0.20 0.00
\sparkspike 0.30 0.00
\sparkspike 0.40 0.10
\sparkspike 0.50 0.00
\definecolor{sparkspikecolor}{named}{red}
\sparkspike 0.60 0.10
\definecolor{sparkspikecolor}{named}{black}
\sparkspike 0.70 0.10
\sparkspike 0.80 0.00
\sparkspike 0.90 0.20
\sparkspike 1.00 0.20
\sparkbottomline
\end{sparkline}
\renewcommand{\sparklineheight}{1.75}}
\\ 
gauss-mix&\begin{minipage}[c][\blankheight]{0pt}\end{minipage}&&\multicolumn{1}{l}{\warmup}&$
\begin{array}{c}
\scriptstyle{132.5} \\[-6pt]
\scriptscriptstyle{(80.5, 171.8)}
\end{array}
$
\noindent\parbox[p]{4ex}{\renewcommand{\sparklineheight}{2.75}
\begin{sparkline}{4}
\sparkspike 0.10 0.30
\sparkspike 0.20 0.00
\definecolor{sparkspikecolor}{named}{red}
\sparkspike 0.30 0.20
\definecolor{sparkspikecolor}{named}{black}
\sparkspike 0.40 0.00
\sparkspike 0.50 0.00
\sparkspike 0.60 0.00
\sparkspike 0.70 0.00
\sparkspike 0.80 0.00
\sparkspike 0.90 0.20
\sparkspike 1.00 0.30
\sparkbottomline
\end{sparkline}
\renewcommand{\sparklineheight}{1.75}}
&$
\begin{array}{c}
\scriptstyle{108.76} \\[-6pt]
\scriptscriptstyle{(66.812, 139.747)}
\end{array}
$
\noindent\parbox[p]{4ex}{\renewcommand{\sparklineheight}{2.75}
\begin{sparkline}{4}
\sparkspike 0.10 0.30
\sparkspike 0.20 0.00
\definecolor{sparkspikecolor}{named}{red}
\sparkspike 0.30 0.20
\definecolor{sparkspikecolor}{named}{black}
\sparkspike 0.40 0.00
\sparkspike 0.50 0.00
\sparkspike 0.60 0.00
\sparkspike 0.70 0.00
\sparkspike 0.80 0.00
\sparkspike 0.90 0.20
\sparkspike 1.00 0.30
\sparkbottomline
\end{sparkline}
\renewcommand{\sparklineheight}{1.75}}
&$
\begin{array}{c}
\scriptstyle{0.79336} \\[-6pt]
\scriptscriptstyle{\pm0.014569}
\end{array}
$
\noindent\parbox[p]{4ex}{\renewcommand{\sparklineheight}{2.75}
\begin{sparkline}{4}
\sparkspike 0.10 0.40
\definecolor{sparkspikecolor}{named}{red}
\sparkspike 0.20 0.10
\definecolor{sparkspikecolor}{named}{black}
\sparkspike 0.30 0.20
\sparkspike 0.40 0.10
\sparkspike 0.50 0.10
\sparkspike 0.60 0.00
\sparkspike 0.70 0.00
\sparkspike 0.80 0.00
\sparkspike 0.90 0.00
\sparkspike 1.00 0.10
\sparkbottomline
\end{sparkline}
\renewcommand{\sparklineheight}{1.75}}
\\ 
log-regression&\begin{minipage}[c][\blankheight]{0pt}\end{minipage}&&\multicolumn{1}{l}{\goodinconsistent \scriptsize($8$\flatc, $2$\warmup)}&$
\begin{array}{c}
\scriptstyle{1.0} \\[-6pt]
\scriptscriptstyle{(1.0, 55.0)}
\end{array}
$
\noindent\parbox[p]{4ex}{\renewcommand{\sparklineheight}{2.75}
\begin{sparkline}{4}
\definecolor{sparkspikecolor}{named}{red}
\sparkspike 0.10 0.80
\definecolor{sparkspikecolor}{named}{black}
\sparkspike 0.20 0.00
\sparkspike 0.30 0.00
\sparkspike 0.40 0.00
\sparkspike 0.50 0.00
\sparkspike 0.60 0.00
\sparkspike 0.70 0.00
\sparkspike 0.80 0.00
\sparkspike 0.90 0.10
\sparkspike 1.00 0.10
\sparkbottomline
\end{sparkline}
\renewcommand{\sparklineheight}{1.75}}
&$
\begin{array}{c}
\scriptstyle{0.00} \\[-6pt]
\scriptscriptstyle{(0.000, 113.354)}
\end{array}
$
\noindent\parbox[p]{4ex}{\renewcommand{\sparklineheight}{2.75}
\begin{sparkline}{4}
\definecolor{sparkspikecolor}{named}{red}
\sparkspike 0.10 0.80
\definecolor{sparkspikecolor}{named}{black}
\sparkspike 0.20 0.00
\sparkspike 0.30 0.00
\sparkspike 0.40 0.00
\sparkspike 0.50 0.00
\sparkspike 0.60 0.00
\sparkspike 0.70 0.00
\sparkspike 0.80 0.10
\sparkspike 0.90 0.00
\sparkspike 1.00 0.10
\sparkbottomline
\end{sparkline}
\renewcommand{\sparklineheight}{1.75}}
&$
\begin{array}{c}
\scriptstyle{2.00297} \\[-6pt]
\scriptscriptstyle{\pm0.033177}
\end{array}
$
\noindent\parbox[p]{4ex}{\renewcommand{\sparklineheight}{2.75}
\begin{sparkline}{4}
\sparkspike 0.10 0.10
\sparkspike 0.20 0.00
\sparkspike 0.30 0.00
\sparkspike 0.40 0.10
\sparkspike 0.50 0.00
\sparkspike 0.60 0.00
\sparkspike 0.70 0.00
\sparkspike 0.80 0.10
\definecolor{sparkspikecolor}{named}{red}
\sparkspike 0.90 0.30
\definecolor{sparkspikecolor}{named}{black}
\sparkspike 1.00 0.40
\sparkbottomline
\end{sparkline}
\renewcommand{\sparklineheight}{1.75}}
\\ 
mnemonics&\begin{minipage}[c][\blankheight]{0pt}\end{minipage}&&\multicolumn{1}{l}{\badinconsistent \scriptsize($9$\warmup, $1$\slowdown)}&$
\begin{array}{c}
\scriptstyle{3.0} \\[-6pt]
\scriptscriptstyle{(3.0, 442.6)}
\end{array}
$
\noindent\parbox[p]{4ex}{\renewcommand{\sparklineheight}{2.75}
\begin{sparkline}{4}
\definecolor{sparkspikecolor}{named}{red}
\sparkspike 0.10 0.80
\definecolor{sparkspikecolor}{named}{black}
\sparkspike 0.20 0.00
\sparkspike 0.30 0.00
\sparkspike 0.40 0.00
\sparkspike 0.50 0.00
\sparkspike 0.60 0.00
\sparkspike 0.70 0.00
\sparkspike 0.80 0.00
\sparkspike 0.90 0.00
\sparkspike 1.00 0.20
\sparkbottomline
\end{sparkline}
\renewcommand{\sparklineheight}{1.75}}
&$
\begin{array}{c}
\scriptstyle{7.00} \\[-6pt]
\scriptscriptstyle{(6.896, 1360.945)}
\end{array}
$
\noindent\parbox[p]{4ex}{\renewcommand{\sparklineheight}{2.75}
\begin{sparkline}{4}
\definecolor{sparkspikecolor}{named}{red}
\sparkspike 0.10 0.80
\definecolor{sparkspikecolor}{named}{black}
\sparkspike 0.20 0.00
\sparkspike 0.30 0.00
\sparkspike 0.40 0.00
\sparkspike 0.50 0.00
\sparkspike 0.60 0.00
\sparkspike 0.70 0.00
\sparkspike 0.80 0.00
\sparkspike 0.90 0.00
\sparkspike 1.00 0.20
\sparkbottomline
\end{sparkline}
\renewcommand{\sparklineheight}{1.75}}
&$
\begin{array}{c}
\scriptstyle{3.08823} \\[-6pt]
\scriptscriptstyle{\pm0.056426}
\end{array}
$
\noindent\parbox[p]{4ex}{\renewcommand{\sparklineheight}{2.75}
\begin{sparkline}{4}
\sparkspike 0.10 0.10
\sparkspike 0.20 0.10
\sparkspike 0.30 0.10
\sparkspike 0.40 0.00
\sparkspike 0.50 0.00
\definecolor{sparkspikecolor}{named}{red}
\sparkspike 0.60 0.30
\definecolor{sparkspikecolor}{named}{black}
\sparkspike 0.70 0.00
\sparkspike 0.80 0.20
\sparkspike 0.90 0.10
\sparkspike 1.00 0.10
\sparkbottomline
\end{sparkline}
\renewcommand{\sparklineheight}{1.75}}
\\ 
naive-bayes&\begin{minipage}[c][\blankheight]{0pt}\end{minipage}&&\multicolumn{1}{l}{\badinconsistent \scriptsize($6$\flatc, $3$\warmup, $1$\slowdown)}&$
\begin{array}{c}
\scriptstyle{1.0} \\[-6pt]
\scriptscriptstyle{(1.0, 668.4)}
\end{array}
$
\noindent\parbox[p]{4ex}{\renewcommand{\sparklineheight}{2.75}
\begin{sparkline}{4}
\definecolor{sparkspikecolor}{named}{red}
\sparkspike 0.10 0.90
\definecolor{sparkspikecolor}{named}{black}
\sparkspike 0.20 0.00
\sparkspike 0.30 0.00
\sparkspike 0.40 0.00
\sparkspike 0.50 0.00
\sparkspike 0.60 0.00
\sparkspike 0.70 0.00
\sparkspike 0.80 0.00
\sparkspike 0.90 0.00
\sparkspike 1.00 0.10
\sparkbottomline
\end{sparkline}
\renewcommand{\sparklineheight}{1.75}}
&$
\begin{array}{c}
\scriptstyle{0.00} \\[-6pt]
\scriptscriptstyle{(0.000, 842.543)}
\end{array}
$
\noindent\parbox[p]{4ex}{\renewcommand{\sparklineheight}{2.75}
\begin{sparkline}{4}
\definecolor{sparkspikecolor}{named}{red}
\sparkspike 0.10 0.90
\definecolor{sparkspikecolor}{named}{black}
\sparkspike 0.20 0.00
\sparkspike 0.30 0.00
\sparkspike 0.40 0.00
\sparkspike 0.50 0.00
\sparkspike 0.60 0.00
\sparkspike 0.70 0.00
\sparkspike 0.80 0.00
\sparkspike 0.90 0.00
\sparkspike 1.00 0.10
\sparkbottomline
\end{sparkline}
\renewcommand{\sparklineheight}{1.75}}
&$
\begin{array}{c}
\scriptstyle{1.25032} \\[-6pt]
\scriptscriptstyle{\pm0.039789}
\end{array}
$
\noindent\parbox[p]{4ex}{\renewcommand{\sparklineheight}{2.75}
\begin{sparkline}{4}
\sparkspike 0.10 0.40
\definecolor{sparkspikecolor}{named}{red}
\sparkspike 0.20 0.10
\definecolor{sparkspikecolor}{named}{black}
\sparkspike 0.30 0.10
\sparkspike 0.40 0.10
\sparkspike 0.50 0.10
\sparkspike 0.60 0.00
\sparkspike 0.70 0.10
\sparkspike 0.80 0.00
\sparkspike 0.90 0.00
\sparkspike 1.00 0.10
\sparkbottomline
\end{sparkline}
\renewcommand{\sparklineheight}{1.75}}
\\ 
neo4j-analytics&\begin{minipage}[c][\blankheight]{0pt}\end{minipage}&&\multicolumn{1}{l}{\goodinconsistent \scriptsize($7$\warmup, $3$\flatc)}&$
\begin{array}{c}
\scriptstyle{64.5} \\[-6pt]
\scriptscriptstyle{(1.0, 1298.2)}
\end{array}
$
\noindent\parbox[p]{4ex}{\renewcommand{\sparklineheight}{2.75}
\begin{sparkline}{4}
\definecolor{sparkspikecolor}{named}{red}
\sparkspike 0.10 0.80
\definecolor{sparkspikecolor}{named}{black}
\sparkspike 0.20 0.00
\sparkspike 0.30 0.00
\sparkspike 0.40 0.00
\sparkspike 0.50 0.00
\sparkspike 0.60 0.00
\sparkspike 0.70 0.00
\sparkspike 0.80 0.00
\sparkspike 0.90 0.00
\sparkspike 1.00 0.20
\sparkbottomline
\end{sparkline}
\renewcommand{\sparklineheight}{1.75}}
&$
\begin{array}{c}
\scriptstyle{256.26} \\[-6pt]
\scriptscriptstyle{(0.000, 4786.530)}
\end{array}
$
\noindent\parbox[p]{4ex}{\renewcommand{\sparklineheight}{2.75}
\begin{sparkline}{4}
\definecolor{sparkspikecolor}{named}{red}
\sparkspike 0.10 0.80
\definecolor{sparkspikecolor}{named}{black}
\sparkspike 0.20 0.00
\sparkspike 0.30 0.00
\sparkspike 0.40 0.00
\sparkspike 0.50 0.00
\sparkspike 0.60 0.00
\sparkspike 0.70 0.00
\sparkspike 0.80 0.00
\sparkspike 0.90 0.00
\sparkspike 1.00 0.20
\sparkbottomline
\end{sparkline}
\renewcommand{\sparklineheight}{1.75}}
&$
\begin{array}{c}
\scriptstyle{3.77132} \\[-6pt]
\scriptscriptstyle{\pm0.334818}
\end{array}
$
\noindent\parbox[p]{4ex}{\renewcommand{\sparklineheight}{2.75}
\begin{sparkline}{4}
\sparkspike 0.10 0.10
\sparkspike 0.20 0.10
\definecolor{sparkspikecolor}{named}{red}
\sparkspike 0.30 0.40
\definecolor{sparkspikecolor}{named}{black}
\sparkspike 0.40 0.00
\sparkspike 0.50 0.00
\sparkspike 0.60 0.20
\sparkspike 0.70 0.10
\sparkspike 0.80 0.00
\sparkspike 0.90 0.00
\sparkspike 1.00 0.10
\sparkbottomline
\end{sparkline}
\renewcommand{\sparklineheight}{1.75}}
\\ 
par-mnemonics&\begin{minipage}[c][\blankheight]{0pt}\end{minipage}&&\multicolumn{1}{l}{\badinconsistent \scriptsize($9$\warmup, $1$\slowdown)}&$
\begin{array}{c}
\scriptstyle{99.0} \\[-6pt]
\scriptscriptstyle{(5.0, 683.8)}
\end{array}
$
\noindent\parbox[p]{4ex}{\renewcommand{\sparklineheight}{2.75}
\begin{sparkline}{4}
\sparkspike 0.10 0.30
\definecolor{sparkspikecolor}{named}{red}
\sparkspike 0.20 0.40
\definecolor{sparkspikecolor}{named}{black}
\sparkspike 0.30 0.10
\sparkspike 0.40 0.00
\sparkspike 0.50 0.00
\sparkspike 0.60 0.00
\sparkspike 0.70 0.00
\sparkspike 0.80 0.00
\sparkspike 0.90 0.00
\sparkspike 1.00 0.20
\sparkbottomline
\end{sparkline}
\renewcommand{\sparklineheight}{1.75}}
&$
\begin{array}{c}
\scriptstyle{227.20} \\[-6pt]
\scriptscriptstyle{(10.168, 1641.911)}
\end{array}
$
\noindent\parbox[p]{4ex}{\renewcommand{\sparklineheight}{2.75}
\begin{sparkline}{4}
\sparkspike 0.10 0.30
\definecolor{sparkspikecolor}{named}{red}
\sparkspike 0.20 0.40
\definecolor{sparkspikecolor}{named}{black}
\sparkspike 0.30 0.10
\sparkspike 0.40 0.00
\sparkspike 0.50 0.00
\sparkspike 0.60 0.00
\sparkspike 0.70 0.00
\sparkspike 0.80 0.00
\sparkspike 0.90 0.00
\sparkspike 1.00 0.20
\sparkbottomline
\end{sparkline}
\renewcommand{\sparklineheight}{1.75}}
&$
\begin{array}{c}
\scriptstyle{2.36998} \\[-6pt]
\scriptscriptstyle{\pm0.102158}
\end{array}
$
\noindent\parbox[p]{4ex}{\renewcommand{\sparklineheight}{2.75}
\begin{sparkline}{4}
\sparkspike 0.10 0.20
\sparkspike 0.20 0.10
\sparkspike 0.30 0.00
\sparkspike 0.40 0.00
\sparkspike 0.50 0.00
\definecolor{sparkspikecolor}{named}{red}
\sparkspike 0.60 0.20
\definecolor{sparkspikecolor}{named}{black}
\sparkspike 0.70 0.10
\sparkspike 0.80 0.00
\sparkspike 0.90 0.10
\sparkspike 1.00 0.30
\sparkbottomline
\end{sparkline}
\renewcommand{\sparklineheight}{1.75}}
\\ 
philosophers&\begin{minipage}[c][\blankheight]{0pt}\end{minipage}&&\multicolumn{1}{l}{\goodinconsistent \scriptsize($7$\warmup, $3$\flatc)}&$
\begin{array}{c}
\scriptstyle{33.5} \\[-6pt]
\scriptscriptstyle{(1.0, 787.3)}
\end{array}
$
\noindent\parbox[p]{4ex}{\renewcommand{\sparklineheight}{2.75}
\begin{sparkline}{4}
\definecolor{sparkspikecolor}{named}{red}
\sparkspike 0.10 0.60
\definecolor{sparkspikecolor}{named}{black}
\sparkspike 0.20 0.10
\sparkspike 0.30 0.00
\sparkspike 0.40 0.00
\sparkspike 0.50 0.10
\sparkspike 0.60 0.00
\sparkspike 0.70 0.00
\sparkspike 0.80 0.10
\sparkspike 0.90 0.00
\sparkspike 1.00 0.10
\sparkbottomline
\end{sparkline}
\renewcommand{\sparklineheight}{1.75}}
&$
\begin{array}{c}
\scriptstyle{39.56} \\[-6pt]
\scriptscriptstyle{(0.000, 944.836)}
\end{array}
$
\noindent\parbox[p]{4ex}{\renewcommand{\sparklineheight}{2.75}
\begin{sparkline}{4}
\definecolor{sparkspikecolor}{named}{red}
\sparkspike 0.10 0.60
\definecolor{sparkspikecolor}{named}{black}
\sparkspike 0.20 0.10
\sparkspike 0.30 0.00
\sparkspike 0.40 0.00
\sparkspike 0.50 0.10
\sparkspike 0.60 0.00
\sparkspike 0.70 0.00
\sparkspike 0.80 0.10
\sparkspike 0.90 0.00
\sparkspike 1.00 0.10
\sparkbottomline
\end{sparkline}
\renewcommand{\sparklineheight}{1.75}}
&$
\begin{array}{c}
\scriptstyle{1.15929} \\[-6pt]
\scriptscriptstyle{\pm0.064406}
\end{array}
$
\noindent\parbox[p]{4ex}{\renewcommand{\sparklineheight}{2.75}
\begin{sparkline}{4}
\sparkspike 0.10 0.10
\sparkspike 0.20 0.10
\sparkspike 0.30 0.00
\definecolor{sparkspikecolor}{named}{red}
\sparkspike 0.40 0.40
\definecolor{sparkspikecolor}{named}{black}
\sparkspike 0.50 0.00
\sparkspike 0.60 0.10
\sparkspike 0.70 0.00
\sparkspike 0.80 0.10
\sparkspike 0.90 0.10
\sparkspike 1.00 0.10
\sparkbottomline
\end{sparkline}
\renewcommand{\sparklineheight}{1.75}}
\\ 
reactors&\begin{minipage}[c][\blankheight]{0pt}\end{minipage}&&\multicolumn{1}{l}{\badinconsistent \scriptsize($6$\flatc, $4$\slowdown)}&$
\begin{array}{c}
\scriptstyle{1.0} \\[-6pt]
\scriptscriptstyle{(1.0, 864.4)}
\end{array}
$
\noindent\parbox[p]{4ex}{\renewcommand{\sparklineheight}{2.75}
\begin{sparkline}{4}
\definecolor{sparkspikecolor}{named}{red}
\sparkspike 0.10 0.60
\definecolor{sparkspikecolor}{named}{black}
\sparkspike 0.20 0.00
\sparkspike 0.30 0.00
\sparkspike 0.40 0.00
\sparkspike 0.50 0.00
\sparkspike 0.60 0.10
\sparkspike 0.70 0.00
\sparkspike 0.80 0.10
\sparkspike 0.90 0.10
\sparkspike 1.00 0.10
\sparkbottomline
\end{sparkline}
\renewcommand{\sparklineheight}{1.75}}
&$
\begin{array}{c}
\scriptstyle{0.00} \\[-6pt]
\scriptscriptstyle{(0.000, 8520.771)}
\end{array}
$
\noindent\parbox[p]{4ex}{\renewcommand{\sparklineheight}{2.75}
\begin{sparkline}{4}
\definecolor{sparkspikecolor}{named}{red}
\sparkspike 0.10 0.60
\definecolor{sparkspikecolor}{named}{black}
\sparkspike 0.20 0.00
\sparkspike 0.30 0.00
\sparkspike 0.40 0.00
\sparkspike 0.50 0.00
\sparkspike 0.60 0.10
\sparkspike 0.70 0.00
\sparkspike 0.80 0.10
\sparkspike 0.90 0.10
\sparkspike 1.00 0.10
\sparkbottomline
\end{sparkline}
\renewcommand{\sparklineheight}{1.75}}
&$
\begin{array}{c}
\scriptstyle{9.96745} \\[-6pt]
\scriptscriptstyle{\pm0.180175}
\end{array}
$
\noindent\parbox[p]{4ex}{\renewcommand{\sparklineheight}{2.75}
\begin{sparkline}{4}
\sparkspike 0.10 0.10
\sparkspike 0.20 0.20
\sparkspike 0.30 0.10
\definecolor{sparkspikecolor}{named}{red}
\sparkspike 0.40 0.10
\definecolor{sparkspikecolor}{named}{black}
\sparkspike 0.50 0.00
\sparkspike 0.60 0.10
\sparkspike 0.70 0.10
\sparkspike 0.80 0.10
\sparkspike 0.90 0.10
\sparkspike 1.00 0.10
\sparkbottomline
\end{sparkline}
\renewcommand{\sparklineheight}{1.75}}
\\ 
rx-scrabble&\begin{minipage}[c][\blankheight]{0pt}\end{minipage}&&\multicolumn{1}{l}{\badinconsistent \scriptsize($7$\warmup, $3$\slowdown)}&$
\begin{array}{c}
\scriptstyle{77.0} \\[-6pt]
\scriptscriptstyle{(76.0, 109.3)}
\end{array}
$
\noindent\parbox[p]{4ex}{\renewcommand{\sparklineheight}{2.75}
\begin{sparkline}{4}
\definecolor{sparkspikecolor}{named}{red}
\sparkspike 0.10 0.60
\definecolor{sparkspikecolor}{named}{black}
\sparkspike 0.20 0.00
\sparkspike 0.30 0.00
\sparkspike 0.40 0.00
\sparkspike 0.50 0.10
\sparkspike 0.60 0.00
\sparkspike 0.70 0.10
\sparkspike 0.80 0.00
\sparkspike 0.90 0.10
\sparkspike 1.00 0.10
\sparkbottomline
\end{sparkline}
\renewcommand{\sparklineheight}{1.75}}
&$
\begin{array}{c}
\scriptstyle{16.21} \\[-6pt]
\scriptscriptstyle{(15.815, 22.622)}
\end{array}
$
\noindent\parbox[p]{4ex}{\renewcommand{\sparklineheight}{2.75}
\begin{sparkline}{4}
\definecolor{sparkspikecolor}{named}{red}
\sparkspike 0.10 0.60
\definecolor{sparkspikecolor}{named}{black}
\sparkspike 0.20 0.00
\sparkspike 0.30 0.00
\sparkspike 0.40 0.00
\sparkspike 0.50 0.10
\sparkspike 0.60 0.00
\sparkspike 0.70 0.10
\sparkspike 0.80 0.10
\sparkspike 0.90 0.00
\sparkspike 1.00 0.10
\sparkbottomline
\end{sparkline}
\renewcommand{\sparklineheight}{1.75}}
&$
\begin{array}{c}
\scriptstyle{0.20222} \\[-6pt]
\scriptscriptstyle{\pm0.001798}
\end{array}
$
\noindent\parbox[p]{4ex}{\renewcommand{\sparklineheight}{2.75}
\begin{sparkline}{4}
\sparkspike 0.10 0.10
\sparkspike 0.20 0.00
\sparkspike 0.30 0.00
\sparkspike 0.40 0.30
\sparkspike 0.50 0.00
\definecolor{sparkspikecolor}{named}{red}
\sparkspike 0.60 0.10
\definecolor{sparkspikecolor}{named}{black}
\sparkspike 0.70 0.10
\sparkspike 0.80 0.10
\sparkspike 0.90 0.20
\sparkspike 1.00 0.10
\sparkbottomline
\end{sparkline}
\renewcommand{\sparklineheight}{1.75}}
\\ 
scala-kmeans&\begin{minipage}[c][\blankheight]{0pt}\end{minipage}&&\multicolumn{1}{l}{\badinconsistent \scriptsize($6$\slowdown, $4$\nosteadystate)}&\begin{minipage}[c][\blankheight]{0pt}\end{minipage}&\begin{minipage}[c][\blankheight]{0pt}\end{minipage}&\begin{minipage}[c][\blankheight]{0pt}\end{minipage}\\ 
scala-stm-bench7&\begin{minipage}[c][\blankheight]{0pt}\end{minipage}&&\multicolumn{1}{l}{\badinconsistent \scriptsize($9$\warmup, $1$\slowdown)}&$
\begin{array}{c}
\scriptstyle{202.5} \\[-6pt]
\scriptscriptstyle{(76.6, 1327.6)}
\end{array}
$
\noindent\parbox[p]{4ex}{\renewcommand{\sparklineheight}{2.75}
\begin{sparkline}{4}
\definecolor{sparkspikecolor}{named}{red}
\sparkspike 0.10 0.50
\definecolor{sparkspikecolor}{named}{black}
\sparkspike 0.20 0.10
\sparkspike 0.30 0.00
\sparkspike 0.40 0.10
\sparkspike 0.50 0.10
\sparkspike 0.60 0.00
\sparkspike 0.70 0.00
\sparkspike 0.80 0.10
\sparkspike 0.90 0.00
\sparkspike 1.00 0.10
\sparkbottomline
\end{sparkline}
\renewcommand{\sparklineheight}{1.75}}
&$
\begin{array}{c}
\scriptstyle{168.98} \\[-6pt]
\scriptscriptstyle{(64.756, 1099.853)}
\end{array}
$
\noindent\parbox[p]{4ex}{\renewcommand{\sparklineheight}{2.75}
\begin{sparkline}{4}
\definecolor{sparkspikecolor}{named}{red}
\sparkspike 0.10 0.50
\definecolor{sparkspikecolor}{named}{black}
\sparkspike 0.20 0.10
\sparkspike 0.30 0.00
\sparkspike 0.40 0.10
\sparkspike 0.50 0.10
\sparkspike 0.60 0.00
\sparkspike 0.70 0.00
\sparkspike 0.80 0.10
\sparkspike 0.90 0.00
\sparkspike 1.00 0.10
\sparkbottomline
\end{sparkline}
\renewcommand{\sparklineheight}{1.75}}
&$
\begin{array}{c}
\scriptstyle{0.80903} \\[-6pt]
\scriptscriptstyle{\pm0.022210}
\end{array}
$
\noindent\parbox[p]{4ex}{\renewcommand{\sparklineheight}{2.75}
\begin{sparkline}{4}
\sparkspike 0.10 0.10
\sparkspike 0.20 0.10
\sparkspike 0.30 0.10
\sparkspike 0.40 0.10
\definecolor{sparkspikecolor}{named}{red}
\sparkspike 0.50 0.20
\definecolor{sparkspikecolor}{named}{black}
\sparkspike 0.60 0.10
\sparkspike 0.70 0.20
\sparkspike 0.80 0.00
\sparkspike 0.90 0.00
\sparkspike 1.00 0.10
\sparkbottomline
\end{sparkline}
\renewcommand{\sparklineheight}{1.75}}
\\ 
scrabble&\begin{minipage}[c][\blankheight]{0pt}\end{minipage}&&\multicolumn{1}{l}{\warmup}&$
\begin{array}{c}
\scriptstyle{11.0} \\[-6pt]
\scriptscriptstyle{(11.0, 22.4)}
\end{array}
$
\noindent\parbox[p]{4ex}{\renewcommand{\sparklineheight}{2.75}
\begin{sparkline}{4}
\definecolor{sparkspikecolor}{named}{red}
\sparkspike 0.10 0.90
\definecolor{sparkspikecolor}{named}{black}
\sparkspike 0.20 0.00
\sparkspike 0.30 0.00
\sparkspike 0.40 0.00
\sparkspike 0.50 0.00
\sparkspike 0.60 0.00
\sparkspike 0.70 0.00
\sparkspike 0.80 0.00
\sparkspike 0.90 0.00
\sparkspike 1.00 0.10
\sparkbottomline
\end{sparkline}
\renewcommand{\sparklineheight}{1.75}}
&$
\begin{array}{c}
\scriptstyle{4.45} \\[-6pt]
\scriptscriptstyle{(4.266, 8.287)}
\end{array}
$
\noindent\parbox[p]{4ex}{\renewcommand{\sparklineheight}{2.75}
\begin{sparkline}{4}
\definecolor{sparkspikecolor}{named}{red}
\sparkspike 0.10 0.90
\definecolor{sparkspikecolor}{named}{black}
\sparkspike 0.20 0.00
\sparkspike 0.30 0.00
\sparkspike 0.40 0.00
\sparkspike 0.50 0.00
\sparkspike 0.60 0.00
\sparkspike 0.70 0.00
\sparkspike 0.80 0.00
\sparkspike 0.90 0.00
\sparkspike 1.00 0.10
\sparkbottomline
\end{sparkline}
\renewcommand{\sparklineheight}{1.75}}
&$
\begin{array}{c}
\scriptstyle{0.33618} \\[-6pt]
\scriptscriptstyle{\pm0.008877}
\end{array}
$
\noindent\parbox[p]{4ex}{\renewcommand{\sparklineheight}{2.75}
\begin{sparkline}{4}
\sparkspike 0.10 0.10
\sparkspike 0.20 0.00
\sparkspike 0.30 0.20
\sparkspike 0.40 0.00
\definecolor{sparkspikecolor}{named}{red}
\sparkspike 0.50 0.20
\definecolor{sparkspikecolor}{named}{black}
\sparkspike 0.60 0.00
\sparkspike 0.70 0.00
\sparkspike 0.80 0.10
\sparkspike 0.90 0.00
\sparkspike 1.00 0.40
\sparkbottomline
\end{sparkline}
\renewcommand{\sparklineheight}{1.75}}
\\ 

\hline
\end{longtable}
}


\bibliographystyle{plain}
\bibliography{bib}

\end{document}
